\documentclass[../../../Main.tex]{subfiles}
\begin{document}
\subsection*{Definition}
If a function $f(x_1,\dots, x_n)$ of $n$ variables $x_1,\dots, x_n$ is such that, for any constant $\lambda$
\begin{equation*}
    f(\lambda x_1,\dots,\lambda x_n )=\lambda^mf( x_1,\dots, x_n )
\end{equation*}
then $f(\lambda x_1,\dots,\lambda x_n )$ is called homogeneous of degree $m$, with $m>1$.

\subsubsection*{Euler’s theorem on homogeneous functions.} The partial de-rivative of homogeneous function obey the relation 
\begin{equation*}
    \sum_{i=1}^{n }x_i\frac{\partial f}{\partial x_i}=mf
\end{equation*} 

\subsection*{Thermodynamic}
Mathematically, extensive properties are homogeneous functions of first order, while intensive properties are homogeneous functions of order zero. 
\subsubsection*{Extensive properties.} an extensive property scales linearly with the system’s size. Properties such as $U$, $V$ , $m$, $n$, and $N$ are all examples of extensive properties; they will double their values upon doubling the size of the system. 

\subsubsection*{Intensive properties.} An intensive property does not depend on the size (or extent) of the system; it is a scale invariant. The ratio between two extensive properties is an intensive property. The molar mass M is therefore an intensive property.
\end{document}