\documentclass[../../../Main.tex]{subfiles}
\begin{document}
State function, such as internal energy \(U \) and entropy \(S \), can be thought as conservative field. The condition that must be satisfied by conservative field $\mathbf{V}$ is
\begin{equation*}
    \nabla\times \mathbf{V}=0
\end{equation*}
Suppose we actually evaluate the curl of vector function $\mathbf{V}(x,y,z)$, we get
\begin{align*}
    \nabla\times \mathbf{V}&=\begin{pmatrix}
        \mathbf{\hat{x}} & \mathbf{\hat{y}} & \mathbf{\hat{z}}\\
        \dfrac{\partial }{\partial x} & \dfrac{\partial }{\partial y} & \dfrac{\partial }{\partial z}\\
        V_x& V_y & V_z
    \end{pmatrix}\\
    \\
    \nabla\times \mathbf{V}&=
    \mathbf{\hat{i}} \biggl(\dfrac{\partial V_z}{\partial y} -\dfrac{\partial V_y}{\partial z}\biggr)+ 
    \mathbf{\hat{j}} \biggl(\dfrac{\partial V_x}{\partial z}-\dfrac{\partial V_z}{\partial x} \biggr) + 
    \mathbf{\hat{k}}\biggl( \dfrac{\partial V_y}{\partial x}-\dfrac{\partial V_x}{\partial y}\biggr)
\end{align*} 
Since \textbf{V}, as a conservative field, has curl of zero, those term inside parenthesis can be evaluated into
\begin{equation*}
    \dfrac{\partial V_z}{\partial y} =\dfrac{\partial V_y}{\partial z},\quad
    \dfrac{\partial V_x}{\partial z}=\dfrac{\partial V_z}{\partial x},\quad
    \dfrac{\partial V_y}{\partial x}=\dfrac{\partial V_x}{\partial y}
\end{equation*}
For state function \(U(S,V,N)\), the equation reads
\begin{equation*}
    \dfrac{\partial U_N}{\partial V} =\dfrac{\partial U_V}{\partial N},\quad
    \dfrac{\partial U_S}{\partial N}=\dfrac{\partial U_N}{\partial S},\quad
    \dfrac{\partial U_V}{\partial S}=\dfrac{\partial U_S}{\partial V}
\end{equation*}
Of course you can't evaluate the curl of state function, but hear me out. What we consider is not the function $U$ itself, but rather, the differential $dU$. Its total differential may be written as 
\begin{equation*}
    dU(S,V,N)=\frac{\partial U}{\partial S}\bigg|_{V,N} dS + \frac{ \partial U}{\partial V}\bigg|_{S,N} dV+ \frac{\partial U}{\partial N}\bigg|_{S,V}dN
\end{equation*}
Here, the differentials $(dS,dT,dN)$ act like unit vector, thus we can pretend that $dU$ is a vector field with components of
\begin{equation*}
    U_S=\frac{\partial U}{\partial S}\bigg|_{V,N}, \quad
    U_V=\frac{ \partial U}{\partial V}\bigg|_{S,N},\quad
    U_N=\frac{\partial U}{\partial N}\bigg|_{S,V}
\end{equation*}
Therefore
\begin{align*}
    \dfrac{\partial }{\partial V} \frac{\partial U}{\partial N}  = \dfrac{\partial }{\partial N}\frac{ \partial U}{\partial V},\quad
    \dfrac{\partial }{\partial N}\frac{\partial U}{\partial S}=\dfrac{\partial }{\partial S}\frac{\partial U}{\partial N},\quad
    \dfrac{\partial }{\partial S}\dfrac{ \partial U}{\partial V}=\dfrac{\partial }{\partial V}\frac{\partial U}{\partial S}
\end{align*}
This is what it means to be an exact differential.

\end{document}