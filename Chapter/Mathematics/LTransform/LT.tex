\documentclass[../../../Main.tex]{subfiles}
\begin{document}
\subsection{Definition}
Consider monotonic function $f(x)$. The slope of $f(x)$ expressed as 
\begin{equation*}
    s(x)=\frac{d}{dx}f(x)
\end{equation*}
Suppose now we want to transform $f(x)$ into function $G(s)$ as function of its slope. We write 
\begin{equation*}
    G(s)=f[x(s)]-sx(s)
\end{equation*}
where $x(s)$ reads as $x$ in terms of its slope $s$. The function $G(s)$ referred as Legendre transform of $f(x)$. If we want to transform $G(s)$ back into $f(x)$, called inverse transform, we write 
\begin{equation*}
    f(x)=G[s(x)]+xs(x)
\end{equation*}

\subsection{Derivation}
Consider the same monotonic function $f(x)$. Say that its tangent line intercept the y-axis at $Q$. Another family of the tangent line of the same slope will also intercept the y-axis; they will, however, did it at different point. We define the intercept of origin $O$ and $Q$ as $G(s)$.

To find the actual intercept, note that line passing through $Q$ has the form $y=mx+Q$. Since we define $G(s )$ as the line $OG$, we have
\begin{equation*}
    G(s)=f[x(s)]-sx(s)
\end{equation*}

The differential of $G$ is 
\begin{equation*}
    dG(s)=\frac{\partial G}{\partial s}ds=-x(s)\;ds
\end{equation*}
Hence
\begin{equation*}
    x=-\frac{dG(s)}{ds}
\end{equation*}
In a sense the symmetry of $f(x)$ and $G(s)$ can be traced to equation above. $f(x)$ has the slope of $s$, where $G(s)$ is $x$. We can therefore write 
\begin{equation*}
    f(x)=G[s(x)]+xs(x)
\end{equation*}
to transform $G(s)$ back to $f(x)$.
 
\begin{figure*}
    \centering
    \normfig{../../../Rss/Mathematics/LegendreTransform/LT.png}
    \caption*{Figure: Geometric interpertation of Legendre transformation}
\end{figure*}

\subsection{Multivariable Function}
\subsubsection{Single variable transform.} Consider some multivariable function, say $f(x,y,z)$ Suppose we want to construct Legendre transform of $f(x,y,z)$ with respect to $x$. First we have the slope
\begin{equation*}
    s=\frac{\partial f}{\partial x}\bigg|_{y,z}
\end{equation*}
We then write the transform of $f(x)$ as 
\begin{equation*}
    \mathcal{L}_x[f(x,y,z)] = f(x(s),y,z)-sx(s)
\end{equation*}
which is the same for single variable function.

\subsubsection{Multivariable transform.} Suppose that, with the same function, we want to perform Legendre transforms with respect to all variable $x,y,z$. First we define
\begin{equation*}
    s=\frac{\partial f}{\partial x}\bigg|_{y,x}\quad ; t=\frac{\partial f}{\partial y}\bigg|_{x,z}\quad ;u=\frac{\partial f}{\partial z}\bigg|_{x,y}
\end{equation*}
We then write
\begin{equation*}
    \mathcal{L}_{x,y,z}[f(x,y,z)] = f[x(s),y(t),z(u)]-sx(s)-ty(t)-uz(u)
\end{equation*}
\end{document}