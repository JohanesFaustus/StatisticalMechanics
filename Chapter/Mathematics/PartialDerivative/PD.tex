\documentclass[../../../Main.tex]{subfiles}
\begin{document}
\subsection*{Identity Involving Partial Derivative}
The Jacobian of $[u(x,y), v(x,y)]$ with respect to $(x, y)$ is defined by
\begin{equation*}
    \frac{\partial (u,v)}{\partial (x,y)}=\begin{vmatrix}
       \dfrac{\partial u}{\partial x} & \dfrac{\partial u}{\partial y}\\
       \\
       \dfrac{\partial v}{\partial x} & \dfrac{\partial v}{\partial y}
    \end{vmatrix}    
\end{equation*}
Here are some indentity relating the Jacobian with partial derivative.

\subsubsection*{Unity.} Unity as in one
\begin{equation*}
    \frac{\partial (u,v)}{\partial (x,y)}=1
\end{equation*}

\emph{Proof.} Trivial
\begin{equation*}
    \frac{\partial (x,y)}{\partial (x,y)}=\begin{vmatrix}
        \dfrac{\partial x}{\partial x} & \dfrac{\partial x}{\partial y}\\
        \\
        \dfrac{\partial y}{\partial x} & \dfrac{\partial y}{\partial y}
     \end{vmatrix}=\dfrac{\partial x}{\partial x} \dfrac{\partial y}{\partial y}-   \dfrac{\partial x}{\partial y}\dfrac{\partial y}{\partial x}  =1\quad\blacksquare
\end{equation*}

\subsubsection*{Change of order.} It can be proved that change of order cost the minus sign 
\begin{equation*}
    \frac{\partial (u,v)}{\partial (x,y)}=- \frac{\partial (v,u)}{\partial (x,y)}= -\frac{\partial (u,v)}{\partial (y,x)}
\end{equation*}

\emph{Proof.} Those three terms literally have the same value when evaluated
\begin{align*}
    \frac{\partial (u,v)}{\partial (x,y)}&=\begin{vmatrix}
        \dfrac{\partial u}{\partial x} & \dfrac{\partial u}{\partial y}\\
        \\
        \dfrac{\partial v}{\partial x} & \dfrac{\partial v}{\partial y}
     \end{vmatrix}  
     =\dfrac{\partial u}{\partial x} \dfrac{\partial v}{\partial y}-  \dfrac{\partial u}{\partial y} \dfrac{\partial v}{\partial x} 
     \\\\
     - \frac{\partial (v,u)}{\partial (x,y)}&=-\begin{vmatrix}
        \dfrac{\partial v}{\partial x} & \dfrac{\partial v}{\partial y}\\
        \\
        \dfrac{\partial u}{\partial x} & \dfrac{\partial u}{\partial y}
     \end{vmatrix} =\dfrac{\partial v}{\partial y}\dfrac{\partial u}{\partial x} -\dfrac{\partial v}{\partial x} \dfrac{\partial u}{\partial y}
     \\\\
     \frac{\partial (u,v)}{\partial (y,x)}&=-\begin{vmatrix}
        \dfrac{\partial u}{\partial y} & \dfrac{\partial u}{\partial x}\\
        \\
        \dfrac{\partial v}{\partial y} & \dfrac{\partial v}{\partial x}
     \end{vmatrix}
     =\dfrac{\partial u}{\partial x}\dfrac{\partial v}{\partial y} -\dfrac{\partial u}{\partial y}\dfrac{\partial v}{\partial x}
\end{align*}
See? $\blacksquare$

\subsubsection*{Jacobian.} In terms of Jacobian, partial derivative of $u$ with respect to $x$ can be written as 
\begin{equation*}
    \frac{\partial u}{\partial x}\bigg|_y\;=\frac{\partial (u,y)}{\partial (x,y)} 
\end{equation*}

\emph{Proof.} Just evaluate the Jacobian
\begin{equation*}
    \frac{\partial (u,y)}{\partial (x,y)}=\begin{vmatrix}
        \dfrac{\partial u}{\partial x} & \dfrac{\partial u}{\partial y}\\
        \\
        \dfrac{\partial y}{\partial x} & \dfrac{\partial y}{\partial y}
     \end{vmatrix}   
     =\dfrac{\partial u}{\partial x} \dfrac{\partial y}{\partial y}-\dfrac{\partial u}{\partial y} \dfrac{\partial y}{\partial x} =\dfrac{\partial u}{\partial x}\quad \blacksquare
\end{equation*}

\subsubsection*{Chain rule for partial derivative.} The expression is
\begin{equation*}
    \frac{\partial (u,y)}{\partial (x,y)}= \frac{\partial (u,y)}{\partial (w,z)}\frac{\partial (w,z)}{\partial (x,y)}
\end{equation*}

\emph{Proof.} The total differential of $u$ and $v$ as function $w$ and $z$ read
\begin{equation*}
    du=\frac{\partial u}{\partial w}\;dw+ \frac{\partial u}{\partial v}\;dz\quad\land\quad
    dv=\frac{\partial v}{\partial w}\;dw+ \frac{\partial v}{\partial z}\;dz
\end{equation*}
We can therefore evaluate the Jacobian 
\begin{align*}
    \frac{\partial (u,y)}{\partial (x,y)}&=\begin{vmatrix}
        \dfrac{\partial u}{\partial x} & \dfrac{\partial u}{\partial y}
        \\\\
        \dfrac{\partial v}{\partial x} & \dfrac{\partial v}{\partial y}
     \end{vmatrix}=
     \begin{vmatrix}
        \dfrac{\partial u}{\partial w} \dfrac{\partial w}{\partial x}+ \dfrac{\partial u}{\partial z} \dfrac{\partial z}{\partial x}  & \dfrac{\partial u}{\partial w}\dfrac{\partial w}{\partial y} + \dfrac{\partial u}{\partial z} \dfrac{\partial z}{\partial y}  
        \\\\
        \dfrac{\partial v}{\partial w}\dfrac{\partial w}{\partial x} + \dfrac{\partial v}{\partial z} \dfrac{\partial z}{\partial x}   & \dfrac{\partial v}{\partial w}\dfrac{\partial w}{\partial y} +\dfrac{\partial v}{\partial z} \dfrac{\partial z}{\partial y} 
     \end{vmatrix}   
     \\\\
     &=\begin{vmatrix}
        \begin{pmatrix}
            \dfrac{\partial u}{\partial w} & \dfrac{\partial u}{\partial z} 
            \\\\
            \dfrac{\partial v}{\partial w}  & \dfrac{\partial v}{\partial z} 
        \end{pmatrix}
        \begin{pmatrix}
            \dfrac{\partial w}{\partial x} & \dfrac{\partial w}{\partial y} 
            \\\\
            \dfrac{\partial z}{\partial x}  & \dfrac{\partial z}{\partial y} 
        \end{pmatrix}
     \end{vmatrix}=
     \begin{vmatrix}
        \dfrac{\partial u}{\partial w} & \dfrac{\partial u}{\partial z} 
        \\\\
        \dfrac{\partial v}{\partial w}  & \dfrac{\partial v}{\partial z} 
     \end{vmatrix}
     \begin{vmatrix}
        \dfrac{\partial w}{\partial x} & \dfrac{\partial w}{\partial y} 
        \\\\
        \dfrac{\partial z}{\partial x}  & \dfrac{\partial z}{\partial y} 
     \end{vmatrix}
     \\\\
     \frac{\partial (u,y)}{\partial (x,y)}&=\frac{\partial (u,y)}{\partial (w,z)}\frac{\partial (w,z)}{\partial (x,y)} \quad \blacksquare
\end{align*}

\subsubsection*{The real chain rule.} We have 
\begin{equation*}
    \frac{\partial x}{\partial z}\bigg|_y\;\frac{\partial z}{\partial x}\bigg|_y\;=1
\end{equation*}

\emph{Proof.} Trivial
\begin{equation*}
    1=\frac{\partial (x,y)}{\partial (x,y)}=\frac{\partial (x,y)}{\partial (z,y)}\frac{\partial (z,y)}{\partial (x,y)}=\frac{\partial x}{\partial z}\bigg|_y\;\frac{\partial z}{\partial x}\bigg|_y\; \quad\blacksquare
\end{equation*}

\subsubsection*{Yet another chain rule\dots} Even more chain rule\dots
\begin{equation*}
    \frac{\partial x}{\partial y}\bigg|_w\;=\frac{\partial x}{\partial z}\bigg|_w\;\frac{\partial z}{\partial y}\bigg|_w
\end{equation*}
\emph{Proof.} Trivial
\begin{equation*}
    \frac{\partial x}{\partial y}\bigg|_w\;=\frac{\partial (x,w)}{\partial (y,w)} =\frac{\partial (x,w)}{\partial (z,w)}\frac{\partial (z,w)}{\partial (y,w)}  =\frac{\partial x}{\partial z}\bigg|_w\;\frac{\partial z}{\partial y}\bigg|_w
\end{equation*}

\subsubsection*{Cyclic rule.}This is chain rule all over again\dots
\begin{equation*}
    \frac{\partial x}{\partial z}\bigg|_y \;\frac{\partial z}{\partial y}\bigg|_x \; \frac{\partial y}{\partial x}\bigg|_z\;=-1
\end{equation*}

\emph{Proof.} Trivial 
\begin{multline*}
    1=\frac{\partial (x,y)}{\partial (x,y)}=\frac{\partial (x,y)}{\partial (z,y)} \frac{\partial (z,y)}{\partial (z,x)} \frac{\partial (z,x)}{\partial (x,y)}
    =-\frac{\partial (x,y)}{\partial (z,y)} \frac{\partial (y,z)}{\partial (x,z)} \frac{\partial (z,x)}{\partial (y,x)}\\
    =-\frac{\partial x}{\partial z}\bigg|_y\;\frac{\partial y}{\partial x}\bigg|_z\;\frac{\partial z}{\partial y}\bigg|_x \;\quad\blacksquare
\end{multline*}

\subsection*{Application in Thermodynamics}
Here we will derive some usefull intensive parameter used in thermodynamics. We assumed entopy function $S$ has the form of 
\begin{equation*}
    S=S(U, V, N_{i|r} )
\end{equation*}
where $N$ is number of chemical potential and $N_{i|r}\equiv N_1,\dots N_r$. Therefore, its total differential is 
\begin{equation*}
    dS=\frac{\partial S}{\partial U}\bigg|_{V, N_{i|r} }dU + \frac{\partial S}{ \partial V}\bigg|_{U, N_{i|r} }dV +\sum_{j=1}^{r}\frac{\partial S}{ \partial N_j}\bigg|_{U,V, N_{i\neq r} }dN_j
\end{equation*}
We also assume the following quantities
\begin{equation*}
    T=\frac{\partial U}{\partial S}\bigg|_{V, N_{i} }\quad
    ;P=-\frac{\partial U}{\partial V}\bigg|_{S, N_{i} }\quad
    ;\mu_j=\frac{\partial U}{\partial S}\bigg|_{S, V, N_{i\neq j} }
\end{equation*}

\subsubsection*{First identity.} As follows
\begin{equation*}
    \frac{\partial S}{\partial U}\bigg|_{V, N_i}=\frac{1}{T}
\end{equation*}

\emph{Proof.} We use chain rule with $x\rightarrow U,  y \rightarrow V ,z \rightarrow S$; while keeping all the $N_i$ constant
\begin{equation*}
    \frac{\partial U}{\partial S}\bigg|_{V,N_i}\;\frac{\partial S}{\partial U}\bigg|_{V,N_i}=1\implies
    \frac{\partial S}{\partial U}\bigg|_{V,N_i}=\biggl(\frac{\partial U}{\partial S}\bigg|_{V,N_i}\biggr)^{-1}
\end{equation*}
Then, from the definition of temperature
\begin{equation*}
    \frac{\partial S}{\partial U}\bigg|_{V,N_i}=\frac{1}{T}\quad \blacksquare
\end{equation*}

\subsubsection*{Second identity.} The identity written as
\begin{equation*}
    \frac{\partial S}{\partial V}\bigg|_{U, N_i}=\frac{P}{T}
\end{equation*}

\emph{Proof.} We invoke cyclic rule with $x \rightarrow U , y \rightarrow V , z \rightarrow S$; while keeping all the $N_i$ constant
\begin{equation*}
   1= -\frac{\partial U}{\partial S}\bigg|_{V,N_{i}} \;\frac{\partial S}{\partial V}\bigg|_{U,N_{i}} \; \frac{\partial V}{\partial U}\bigg|_{U,N_{i}}
\end{equation*}
Then, from the first identity and the definition of pressure
\begin{equation*}
   1= T\;\frac{\partial S}{\partial V}\bigg|_{U,N_{i}} \;\frac{1}{P}\implies
   \frac{\partial S}{\partial V}\bigg|_{U,N_{i}} =\frac{P}{T}\quad \blacksquare
\end{equation*}

\subsubsection*{Third Identity.} Expressed as 
\begin{equation*}
    \frac{\partial S}{\partial N_j}\bigg|_{U, N_{i\neq j}}=-\frac{P}{T}
\end{equation*}

\emph{Proof.} We again invoke cyclic with $x \rightarrow U , y \rightarrow N j , z \rightarrow S$; while keeping $V$ and all $N$ except $N_{i}$ constant
\begin{equation*}
   1= -\frac{\partial U}{\partial S}\bigg|_{V,N_{i}} \;\frac{\partial S}{\partial N_j}\bigg|_{U,N_{i\neq j}} \; \frac{\partial N_j}{\partial U}\bigg|_{U,N_{i\neq j}}
\end{equation*}
Then, from the definition of temperature and chemical potential
\begin{equation*}
    1= -T\frac{\partial S}{\partial N_j}\bigg|_{U,N_{i\neq j}} \frac{1}{\mu_j}\implies
    \frac{\partial S}{\partial N_j}\bigg|_{U,N_{i\neq j}}=-\frac{\mu_j}{T}\quad \blacksquare
 \end{equation*}

\end{document}