\documentclass[../../../Main.tex]{subfiles}
\begin{document}
\subsection*{Bose Statistics}
Bose derived Planck's law independent of classical electrodynamics to obtain coefficient $8\pi \nu^2/c^3$. He considered gas of photon with energy of $E=h\nu$ and momentum $p=hv/c$. He assumed that quantum state is described by space and momentum phase space. Next he determined the volume of phase space within interval $(\nu,\nu+d\nu)$, which is
\begin{equation*}
    d\omega=V\;d\mathcal{V}
\end{equation*}
where $V$ is the spatial volume and $d\mathcal{V}$ is the momentum volume of the shell within radii $h(\nu+d\nu)/c,h\nu$. The said shell volume is given by
\begin{multline*}
    d\mathcal{V}=(\mathcal{V}+d\mathcal{V})-\mathcal{V}=\frac{4\pi h^3}{3c^3}\left[(\nu+d\nu)^3-\nu^3\right] =\\ \frac{4\pi h^3}{3c^3}\left[3\nu^2\;d\nu+3\nu\;d^2\nu+d^3\nu\right]=\frac{4\pi h^3}{c^3}\nu^2\;d\nu
\end{multline*}
The resulting volume of phase space $d\omega$ is
\begin{equation*}
    d\omega=\frac{4\pi V h^3}{c^3}\nu^2\;d\nu
\end{equation*}
This volume then divided by Bose into cells of volume $h^3$. The number $P_\nu$ as phase space cells available to the photon within $(\nu,\nu+d\nu)$ expressed by
\begin{equation*}
    P_\nu=\frac{2}{c^3}d\omega=\frac{8\pi V\nu^2}{c^3}\;d\nu
\end{equation*}
where the factor of 2 comes after taking account the two direction of polarization. 

Next he evaluates the entropy of said system. To do that, he uses the same method as Boltzmann. He defined the cell $\mathcal{P}_v$ as a box which photon are distributed. Let $n_{k\nu}$ defined as the number of boxes that contain $k$ photon of frequency $\nu$. We have then the following constraints.
\begin{equation*}
    P_\nu=\sum_{k=0}^{\infty}n_{k\nu},\quad U_\nu= hv\sum_{k=0}^{\infty}kn_{k\nu}
\end{equation*}
or simply
\begin{equation*}
    P_\nu=\sum_{k=0}^{\infty}n_{k\nu},\quad N_\nu= \sum_{k=0}^{\infty} kn_{k\nu}
\end{equation*}
where $N_\nu$ is the number of photon with frequency $\nu$. The number of configuration is given by 
\begin{equation*}
    \mathcal{P}_\nu=P_\nu!\left(\prod_{k=0}^{\infty}n_{k\nu}\right)^{-1}
\end{equation*}  
All of those equations applies for distinct frequency $\nu$, what we what however distribution over all frequency $(0,\infty)$. The expression for constraints is
\begin{equation*}
    P_\nu=\sum_{k=0}^{\infty}n_{k\nu},\quad U= h\sum_{k=0}v\sum_{k=0}^{\infty}kn_{k\nu}
\end{equation*}
and the expression for the number of configuration is 
\begin{equation*}
    \mathcal{P}=\sum_{\nu=0}^{\infty}\mathcal{P}_\nu
\end{equation*} 
As usual, the equation that we want to maximize not the configuration $\mathcal{P}$ itself, but rather its logarithm; which may be written as 
\begin{equation*}
    \ln \mathcal{P}=\sum_{\nu=0}^{\infty}\ln\left[\mathcal{P}_v!\left( \prod_{k=0}^{\infty} n_{k\nu}\right)^{-1}\right]
\end{equation*}
On using Stirling's approximation
\begin{align*}
    \ln \mathcal{P}&=\sum_\nu^\infty\left[\mathcal{P}_\nu\ln \left(\mathcal{P}_\nu\right)-\mathcal{P}_\nu - \sum_{k}^{\infty}\left\{n_{k\nu}\ln \left(n_{k\nu} \right)-n_{k\nu} \right\}\right]\\
    \ln \mathcal{P}&=\sum_\nu^\infty\left[\mathcal{P}_\nu\ln \left(\mathcal{P}_\nu\right) - \sum_{k}^{\infty}n_{k\nu}\ln \left(n_{k\nu} \right)\right]
\end{align*}
Since we want to maximize said logarithm with respect to $n_{k\nu}$, we construct the following function using Lagrange's method.
\begin{equation*}
    F(n_{k\nu})=\sum_\nu \left[\mathcal{P}\ln \mathcal{P}_\nu - \sum_{k} n_{k\nu}\ln n_{k\nu} \right]+ 
    \lambda_1\sum_{k}n_{k\nu}+
    \lambda_2h\sum_{v,k}vkn_{k\nu}
\end{equation*}
Setting its derivative to zero
\begin{equation*}
    \frac{dF}{dn_{k\nu}}=-\ln(n_{k\nu})-1+\lambda_1+\lambda_2hk\nu=0
\end{equation*}
which implies
\begin{equation*}
    n_{k\nu}=\exp\left(1-\lambda_1-\lambda_2hk\nu\right)=C\exp\left(-\lambda_2hk\nu\right)
\end{equation*}
Substituting the result into $\mathcal{P}_\nu$ constraints
\begin{equation*}
    \mathcal{P}_\nu=C\sum_{k } \exp\left(-\lambda_2hk\nu\right) \implies C=\mathcal{P}_\nu\left[\sum_{k } \exp\left(-\lambda_2hk\nu\right)\right]^{-1}
\end{equation*}
The term inside parenthesis is a geometric series with ratio of $\exp(-\lambda_2h\nu)$. The constant then can be simply evaluated into 
\begin{equation*}
    C=\mathcal{P}_\nu\left[\frac{1}{1-\exp(-\lambda_2h\nu)}\right]^{-1} = \mathcal{P}_\nu\left[1-\exp(-\lambda_2h\nu)\right]
\end{equation*}
Hence the $n_{k\nu}$ assumes the form 
\begin{equation*}
    n_{k\nu}=\mathcal{P}_\nu\left[1-\exp(-\lambda_2h\nu)\right]\exp\left(-\lambda_2hk\nu\right)
\end{equation*}
On using this to the expression for logarithm of $\mathcal{P}$, we obtain
\begin{multline*}
    \ln \mathcal{P}_{\max}= \sum_\nu \bigg[ \mathcal{P}_\nu \ln \left(\mathcal{P}_\nu\right) \\
    - \sum_{k} n_{k\nu}\ln \{\mathcal{P}_\nu\left[1-\exp(-\lambda_2h\nu)\right]\exp\left(-\lambda_2hk\nu\right) \} \bigg]
\end{multline*}
Then by definition of Boltzmann entropy, we have
\begin{multline*}
    S_B=k_B \ln \mathcal{P}_{\max}= \sum_\nu k_B \bigg[ \mathcal{P}_\nu \ln \left(\mathcal{P}_\nu\right) \\
    - \sum_{k} n_{k\nu}\ln \{\mathcal{P}_\nu\left[1-\exp(-\lambda_2h\nu)\right]\exp\left(-\lambda_2hk\nu\right) \} \bigg]
\end{multline*}
We then write it as such 
\begin{multline*}
    S_B= \sum_\nu k_B \bigg[ \mathcal{P}_\nu \ln \left(\mathcal{P}_\nu\right) \\
    - \sum_{k} n_{k\nu}\left\{\ln(\mathcal{P}_\nu)+\ln (1-\exp\{-\lambda_2h\nu\})-\lambda_2vkh\right\} \bigg]
\end{multline*}
additionally
\begin{equation*}
    S_B= \sum_\nu k_B \bigg[\lambda_2hv\sum_{k}kn_{k\nu} - \sum_{k} n_{k\nu}\ln (1-\exp\{-\lambda_2h\nu\})\bigg]
\end{equation*}
Recall the $\mathcal{P}_\nu$ and $U_\nu$ constraints
\begin{equation*}
    S_B= k_B \bigg[\lambda_2U- \sum_\nu\mathcal{P}_\nu \ln (1-\exp\{-\lambda_2h\nu\})\bigg]
\end{equation*}
Using the following relationship
\begin{equation*}
    \frac{1}{T}=\frac{\partial S}{\partial U}
\end{equation*}
we have
\begin{equation*}
    \frac{1}{T}=k_B\lambda_2\implies \lambda_2=\frac{1}{k_BT}
\end{equation*}
At last, we have 
\begin{equation*}
    n_{k\nu}=\mathcal{P}_\nu\left[1-\exp\left(-\frac{h\nu}{k_BT}\right)\right]\exp\left(-\frac{hk\nu}{k_BT}\right)
\end{equation*}
and
\begin{equation*}
    S_B= \frac{U}{T}- \sum_\nu k_B\mathcal{P}_\nu \ln \left[1-\exp\left(-\frac{h\nu}{k_BT}\right)\right]
\end{equation*}

Henceforward, we use these result to evaluate the following quantity. First we evaluate the number of quanta of frequency $\nu$ denoted as $N_\nu$
\begin{equation*}
    N_\nu=\sum_k^{\infty} kn_{k\nu}
\end{equation*}
Substituting the known value of $n_{k\nu}$, we get 
\begin{align*}
    N_\nu&=\sum_k k \mathcal{P}_\nu\left[1-\exp\left(-\frac{h\nu}{k_BT}\right)\right]\exp\left(-\frac{hk\nu}{k_BT}\right)\\
    &=\mathcal{P}_\nu\left[1-\exp\left(-\frac{h\nu}{k_BT}\right)\right] \sum_k^{\infty} k \exp\left(-\frac{hk\nu}{k_BT}\right)
\end{align*}

To evaluate such sum we consider the geometric series
\begin{equation*}
    \sum_{k=0}^{\infty}r^{k}=\frac{1}{1-r}
\end{equation*}
for $r<1$. Taking a derivative of both sides to get
\begin{equation*}
    \sum_{k=0}^{\infty}kr^{k-1}=\frac{1}{(1-r)^2}
\end{equation*}
Then shift the index down by one 
\begin{equation*}
    \sum_{k=-1}^{\infty}(k+1)r^{k}=\sum_{k=0}^{\infty}(k+1)r^{k}
\end{equation*}
where the left side is allowed since the series is zero at $k=-1$. Then subtract this series with the first series
\begin{equation*}
    \sum_{k=0}^{\infty}(k+1)r^{k}-r^k=\sum_{k=0}^{\infty} kr^k=\frac{1-(1-r)}{(1-r)^2}=\frac{r}{(1-r)^2}
\end{equation*}

Using the formula above, with 
\begin{equation*}
    r=\exp\left(-\frac{h\nu}{k_BT}\right)
\end{equation*}

we can now evaluate the number of quanta
\begin{align*}
    N_\nu&=\mathcal{P}_\nu\left[1-\exp\left(-\frac{h\nu}{k_BT}\right) \right]\exp\left(-\frac{h\nu}{k_BT}\right)\left[1-\exp\left(-\frac{h\nu}{k_BT}\right)\right]^{-2}\\
    &=\mathcal{P}_\nu\exp\left(-\frac{h\nu}{k_BT}\right)\left[1-\exp\left(-\frac{h\nu}{k_BT}\right)\right]^{-1}\\
    &=\mathcal{P}_\nu\left[\exp\left(\frac{h\nu}{k_BT}\right)-1 \right]^{-1}
\end{align*}
\begin{equation*}
    N_\nu=\sum_k kn_{k\nu}=\frac{P_\nu}{\exp(h\nu/k_BT)-1}
\end{equation*}

The second is energy within the same frequency
\begin{equation*}
    U_\nu=h\nu N_\nu =\frac{h\nu P_\nu}{\exp(h\nu/k_BT)-1}
\end{equation*}

\subsection*{Distribution for Indistinguishable Molecules}
The distribution for indistinguishable ideal quantum gas also called the Bose-Einstein distribution. The number particle per unit energy interval is given by 
\begin{equation*}
    n_E=\frac{2\pi}{h^3} \frac{V (2m)^{3/2}\sqrt{E}}{\exp\left[(E-\mu)/k_BT\right]-1}
\end{equation*}

\subsubsection*{Derivation.} Einstein's method has some similarity with Bose's method. He first considers the energy of non-relativistic particle, which is 
\begin{equation*}
    E=\frac{p^2}{2m}
\end{equation*}
This implies
\begin{equation*}
    p=\sqrt{2mE},\quad dp=\frac{m}{p}dE=\frac{m}{\sqrt{2mE}}dE
\end{equation*}
This will be used to determine the volume of $(\mathbf{r},\mathbf{p}+ d\mathbf{p})$ phase space
\begin{multline*}
    d\omega=V \frac{4}{3}\pi\left[(p+dp)^3-p^3\right] = \frac{4}{3}\pi V\left[3p^2\;dp+3p\;d^2p+d^3p\right]\\
    =4\pi Vp^2\;dp =4\pi V \;2mE\;\frac{m}{\sqrt{2mE}}dE=2\pi(2m)^{3/2}\sqrt{E}\;dE
\end{multline*} 
The said volume then divided into cells of $h^3$ volume. Let this number of distinct cells as 
\begin{equation*}
    P_E=\frac{d\omega}{h^3}=\frac{4\pi}{h^3}V(2m)^{3/2}\sqrt{E}\;dE
\end{equation*}
The number of configuration $\mathcal{P}_E$ for distributing $N_E$ indistinguishable particle in $P_E$ boxes is given by 
\begin{equation*}
    \mathcal{P}_E=\frac{(N_E+P_E-1)}{N_E! (P_E-1)!}
\end{equation*}
This gives configuration for particle within $(E,E+dE)$, conversely the total configuration is given by 
\begin{equation*}
    \mathcal{P}=\prod_{E=0}^{\infty}\mathcal{P}_E
\end{equation*}

We then move to determining the entropy $S=k_B \ln \mathcal{P}_{\max}$ by maximizing the logarithm of $\mathcal{P}$ with respect to $N_E$ subjected to the following constraints.
\begin{equation*}
    N=\sum_{E=0}^{\infty}N_E,\quad U=\sum_{E=0}^{\infty}EN_E
\end{equation*}
Using Stirling's approximation on said logarithm, we have 
\begin{align*}
    \ln \mathcal{P}&=\sum_{E}\ln\frac{(N_E+P_E-1)}{N_E! (P_E-1)!}\\
    \ln \mathcal{P}&=\sum_E (N_E+P_E)\ln (N_E+P_E)-N_E\ln(N_E)- P_E \ln (P_E)
\end{align*}
By Lagrange's method
\begin{equation*}
    F=\ln \mathcal{P} +\lambda_1N_E +\lambda_2EN_E
\end{equation*}
Setting its derivative to zero
\begin{equation*}
    \ln (N_E+P_E)+1-\ln (N_E)-1+\lambda_1+\lambda_2=0
\end{equation*}
Taking the exponential and solving for $N_E$
\begin{align*}
    \frac{N_E+P_E}{N_E}&=\exp(-\lambda_1-\lambda_2E)\\
    N_E&=\frac{P_E}{\exp(-\lambda_1-\lambda_2E)-1}
\end{align*}
Substituting this value into the logarithm of $\mathcal{P}$
\begin{multline*}
    \ln \mathcal{P}_{\max}=\sum_E (N_E+P_E)\left[\ln (P_E)+\ln\left(\frac{\exp(-\lambda_1-\lambda_2E)}{\exp(-\lambda_1-\lambda_2E)-1}\right)\right]\\
    - P_E \ln (P_E)-N_E\left[\ln(P_E)-\ln\left(\exp \{-\lambda_1-\lambda_2E\}-1\right)\right] 
\end{multline*}
then 
\begin{multline*}
    \ln \mathcal{P}_{\max}=\sum_E (N_E+P_E)\ln\left(\frac{\exp(-\lambda_1-\lambda_2E)}{\exp(-\lambda_1-\lambda_2E)-1}\right)\\
    + N_E\ln\left(\exp \{-\lambda_1-\lambda_2E\}-1\right)
\end{multline*}
moreover
\begin{equation*}
    \ln \mathcal{P}_{\max}=\sum_E N_E \left(-\lambda_1-\lambda_2E\right)
    + P_E\ln\left(\frac{\exp(-\lambda_1-\lambda_2E)}{\exp(-\lambda_1-\lambda_2E)-1}\right)
\end{equation*}
Hence
\begin{equation*}
    S=\sum_E k_B P_E\ln\left(\frac{\exp(-\lambda_1-\lambda_2E)}{\exp(-\lambda_1-\lambda_2E)-1}\right)-k_B(N\lambda_1+U\lambda_2)
\end{equation*}
On using the following relations
\begin{equation*}
    \frac{1}{T}=\frac{\partial S}{\partial U}\bigg|_{V,N},\quad -\frac{\mu}{T}=\frac{\partial S}{\partial N}\bigg|_{U,V}
\end{equation*}
we have 
\begin{align*}
    \frac{1}{T}&=-\lambda_2k_B\\
    \lambda_2&=-\frac{1}{k_BT}
\end{align*}
and
\begin{align*}
    -\frac{\mu}{T}&=\lambda_1k_B\\
    \lambda_1&=\frac{\mu}{k_BT}
\end{align*}
Thus
\begin{equation*}
    N_E=\frac{2\pi}{h^3} \frac{V (2m)^{3/2}\sqrt{E}}{\exp\left[(E-\mu)/k_BT\right]-1} \;dE
\end{equation*}
The quantity $n_E$ is defined as the number of molecules per unit energy interval
\begin{equation*}
    n_E=\frac{N_E}{dE}=     n_E=\frac{2\pi}{h^3} \frac{V (2m)^{3/2}\sqrt{E}}{\exp\left[(E-\mu)/k_BT\right]-1} \quad\blacksquare
\end{equation*}
\end{document}