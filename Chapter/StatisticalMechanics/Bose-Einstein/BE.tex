\documentclass[../../../Main.tex]{subfiles}
\begin{document}
\subsection*{Bose Statistics}
Bose derived Planck's law independent of classical electrodynamics to obtain coefficient $8\pi \nu^2/c^3$. He considered gas of photon with energy of $E=h\nu$ and momentum $p=hv/c$. He assumed that quantum state is described by space and momentum phase space. Next he determined the volume of phase space within interval $(\nu,\nu+d\nu)$, which is
\begin{equation*}
    d\omega=V\;d\mathcal{V}
\end{equation*}
where $V$ is the spatial volume and $d\mathcal{V}$ is the momentum volume of the shell within radii $h(\nu+d\nu)/c,h\nu$. The said shell volume is given by
\begin{multline*}
    d\mathcal{V}=(\mathcal{V}+d\mathcal{V})-\mathcal{V}=\frac{4\pi h^3}{3c^3}\left[(\nu+d\nu)^3-\nu^3\right] \\= \frac{4\pi h^3}{3c^3}\left[3\nu^2\;d\nu+3\nu\;d^2\nu+d^3\nu\right]=\frac{4\pi h^3}{c^3}\nu^2\;d\nu
\end{multline*}
The resulting volume of phase space $d\omega$ is
\begin{equation*}
    d\omega=\frac{4\pi V h^3}{c^3}\nu^2\;d\nu
\end{equation*}
This volume then divided by Bose into cells of volume $h^3$. The number $P_\nu$ as phase space cells available to the photon within $(\nu,\nu+d\nu)$ expressed by
\begin{equation*}
    P_\nu=\frac{2}{c^3}d\omega=\frac{8\pi V\nu^2}{c^3}\;d\nu
\end{equation*}
where the factor of 2 comes after taking account the two direction of polarization. 

Next he evaluates the entropy of said system. To do that, he uses the same method as Boltzmann. He defined the cell $\mathcal{P}_v$ as a box which photon are distributed. Let $n_{k\nu}$ defined as the number of boxes that contain $k$ photon of frequency $\nu$. We have then the following constraints.
\begin{equation*}
    P_\nu=\sum_{k=0}^{\infty}n_{k\nu},\quad U_\nu= hv\sum_{k=0}^{\infty}kn_{k\nu}
\end{equation*}
or simply
\begin{equation*}
    P_\nu=\sum_{k=0}^{\infty}n_{k\nu},\quad N_\nu= \sum_{k=0}^{\infty} kn_{k\nu}
\end{equation*}
where $N_\nu$ is the number of photon with frequency $\nu$. The number of configuration is given by 
\begin{equation*}
    \mathcal{P}_\nu=P_\nu!\left(\prod_{k=0}^{\infty}n_{k\nu}\right)^{-1}
\end{equation*}  
All of those equations applies for distinct frequency $\nu$, what we what however distribution over all frequency $(0,\infty)$. The expression for constraints is
\begin{equation*}
    P_\nu=\sum_{k=0}^{\infty}n_{k\nu},\quad U= h\sum_{k=0}v\sum_{k=0}^{\infty}kn_{k\nu}
\end{equation*}
and the expression for the number of configuration is 
\begin{equation*}
    \mathcal{P}=\sum_{\nu=0}^{\infty}\mathcal{P}_\nu
\end{equation*} 
As usual, the equation that we want to maximize not the configuration $\mathcal{P}$ itself, but rather its logarithm; which may be written as 
\begin{equation*}
    \ln \mathcal{P}=\sum_{\nu=0}^{\infty}\ln\left[\mathcal{P}_v!\left( \prod_{k=0}^{\infty} n_{k\nu}\right)^{-1}\right]
\end{equation*}
On using Stirling's approximation
\begin{align*}
    \ln \mathcal{P}&=\sum_\nu^\infty\left[\mathcal{P}_\nu\ln \left(\mathcal{P}_\nu\right)-\mathcal{P}_\nu - \sum_{k}^{\infty}\left\{n_{k\nu}\ln \left(n_{k\nu} \right)-n_{k\nu} \right\}\right]\\
    \ln \mathcal{P}&=\sum_\nu^\infty\left[\mathcal{P}_\nu\ln \left(\mathcal{P}_\nu\right) - \sum_{k}^{\infty}n_{k\nu}\ln \left(n_{k\nu} \right)\right]
\end{align*}
Since we want to maximize said logarithm with respect to $n_{k\nu}$, we construct the following function using Lagrange's method.
\begin{equation*}
    F(n_{k\nu})=\sum_\nu \left[\mathcal{P}\ln \mathcal{P}_\nu - \sum_{k} n_{k\nu}\ln n_{k\nu} \right]+ 
    \lambda_1\sum_{k}n_{k\nu}+
    \lambda_2h\sum_{v,k}vkn_{k\nu}
\end{equation*}
Setting its derivative to zero
\begin{equation*}
    \frac{dF}{dn_{k\nu}}=-\ln(n_{k\nu})-1+\lambda_1+\lambda_2hk\nu=0
\end{equation*}
which implies
\begin{equation*}
    n_{k\nu}=\exp\left(1-\lambda_1-\lambda_2hk\nu\right)=C\exp\left(-\lambda_2hk\nu\right)
\end{equation*}
Substituting the result into $\mathcal{P}_\nu$ constraints
\begin{equation*}
    \mathcal{P}_\nu=C\sum_{k } \exp\left(-\lambda_2hk\nu\right) \implies C=\mathcal{P}_\nu\left[\sum_{k } \exp\left(-\lambda_2hk\nu\right)\right]^{-1}
\end{equation*}

The term inside parenthesis is a geometric series with ratio of $\exp(-\lambda_2h\nu)$. The constant then can be simply evaluated into 
\begin{equation*}
    C=\mathcal{P}_\nu\left[\frac{1}{1-\exp(-\lambda_2h\nu)}\right]^{-1} = \mathcal{P}_\nu\left[1-\exp(-\lambda_2h\nu)\right]
\end{equation*}
Hence the $n_{k\nu}$ assumes the form 
\begin{equation*}
    n_{k\nu}=\mathcal{P}_\nu\left[1-\exp(-\lambda_2h\nu)\right]\exp\left(-\lambda_2hk\nu\right)
\end{equation*}
On using this to the expression for logarithm of $\mathcal{P}$, we obtain
\begin{multline*}
    \ln \mathcal{P}_{\max}= \sum_\nu \bigg[ \mathcal{P}_\nu \ln \left(\mathcal{P}_\nu\right) \\
    - \sum_{k} n_{k\nu}\ln \{\mathcal{P}_\nu\left[1-\exp(-\lambda_2h\nu)\right]\exp\left(-\lambda_2hk\nu\right) \} \bigg]
\end{multline*}
Then by definition of Boltzmann entropy, we have
\begin{multline*}
    S_B=k_B \ln \mathcal{P}_{\max}= \sum_\nu k_B \bigg[ \mathcal{P}_\nu \ln \left(\mathcal{P}_\nu\right) \\
    - \sum_{k} n_{k\nu}\ln \{\mathcal{P}_\nu\left[1-\exp(-\lambda_2h\nu)\right]\exp\left(-\lambda_2hk\nu\right) \} \bigg]
\end{multline*}
We then write it as such 
\begin{multline*}
    S_B= \sum_\nu k_B \bigg[ \mathcal{P}_\nu \ln \left(\mathcal{P}_\nu\right) \\
    - \sum_{k} n_{k\nu}\left\{\ln(\mathcal{P}_\nu)+\ln (1-\exp\{-\lambda_2h\nu\})-\lambda_2vkh\right\} \bigg]
\end{multline*}
additionally
\begin{equation*}
    S_B= \sum_\nu k_B \bigg[\lambda_2hv\sum_{k}kn_{k\nu} - \sum_{k} n_{k\nu}\ln (1-\exp\{-\lambda_2h\nu\})\bigg]
\end{equation*}
Recall the $\mathcal{P}_\nu$ and $U_\nu$ constraints
\begin{equation*}
    S_B= k_B \bigg[\lambda_2U- \sum_\nu\mathcal{P}_\nu \ln (1-\exp\{-\lambda_2h\nu\})\bigg]
\end{equation*}
Using the following relationship
\begin{equation*}
    \frac{1}{T}=\frac{\partial S}{\partial U}
\end{equation*}
we have
\begin{equation*}
    \frac{1}{T}=k_B\lambda_2\implies \lambda_2=\frac{1}{k_BT}
\end{equation*}
At last, we have 
\begin{equation*}
    n_{k\nu}=\mathcal{P}_\nu\left[1-\exp\left(-\frac{h\nu}{k_BT}\right)\right]\exp\left(-\frac{hk\nu}{k_BT}\right)
\end{equation*}
and
\begin{equation*}
    S_B= \frac{U}{T}- \sum_\nu k_B\mathcal{P}_\nu \ln \left[1-\exp\left(-\frac{h\nu}{k_BT}\right)\right]
\end{equation*}

Henceforward, we use these result to evaluate the following quantity. First we evaluate the number of quanta of frequency $\nu$ denoted as $N_\nu$
\begin{equation*}
    N_\nu=\sum_k^{\infty} kn_{k\nu}
\end{equation*}
Substituting the known value of $n_{k\nu}$, we get 
\begin{align*}
    N_\nu&=\sum_k k \mathcal{P}_\nu\left[1-\exp\left(-\frac{h\nu}{k_BT}\right)\right]\exp\left(-\frac{hk\nu}{k_BT}\right)\\
    &=\mathcal{P}_\nu\left[1-\exp\left(-\frac{h\nu}{k_BT}\right)\right] \sum_k^{\infty} k \exp\left(-\frac{hk\nu}{k_BT}\right)
\end{align*}

To evaluate such sum we consider the geometric series
\begin{equation*}
    \sum_{k=0}^{\infty}r^{k}=\frac{1}{1-r}
\end{equation*}
for $r<1$. Taking a derivative of both sides to get
\begin{equation*}
    \sum_{k=0}^{\infty}kr^{k-1}=\frac{1}{(1-r)^2}
\end{equation*}
Then shift the index down by one 
\begin{equation*}
    \sum_{k=-1}^{\infty}(k+1)r^{k}=\sum_{k=0}^{\infty}(k+1)r^{k}
\end{equation*}
where the left side is allowed since the series is zero at $k=-1$. Then subtract this series with the first series
\begin{equation*}
    \sum_{k=0}^{\infty}(k+1)r^{k}-r^k=\sum_{k=0}^{\infty} kr^k=\frac{1-(1-r)}{(1-r)^2}=\frac{r}{(1-r)^2}
\end{equation*}

Using the formula above, with 
\begin{equation*}
    r=\exp\left(-\frac{h\nu}{k_BT}\right)
\end{equation*}

we can now evaluate the number of quanta
\begin{align*}
    N_\nu&=\mathcal{P}_\nu\left[1-\exp\left(-\frac{h\nu}{k_BT}\right) \right]\exp\left(-\frac{h\nu}{k_BT}\right)\left[1-\exp\left(-\frac{h\nu}{k_BT}\right)\right]^{-2}\\
    &=\mathcal{P}_\nu\exp\left(-\frac{h\nu}{k_BT}\right)\left[1-\exp\left(-\frac{h\nu}{k_BT}\right)\right]^{-1}\\
    &=\mathcal{P}_\nu\left[\exp\left(\frac{h\nu}{k_BT}\right)-1 \right]^{-1}
\end{align*}
\begin{equation*}
    N_\nu=\sum_k kn_{k\nu}=\frac{\nu^2}{c^3}\frac{8\pi V}{\exp(h\nu/k_BT)-1}\;d\nu
\end{equation*}

The second is energy within the same frequency
\begin{equation*}
    U_\nu=h\nu N_\nu =\frac{\nu^3}{c^3}\frac{8\pi h V}{\exp(h\nu/k_BT)-1}\;d\nu
\end{equation*}
To obtain the average energy per unit frequency, we divide $U_\nu$ by spatial volume (since we derived it from phase space volume) $V$ and by unit frequency $d\nu$
\begin{equation*}
    \mathcal{U}(\nu,T)=\frac{U_\nu}{V\;d\nu}=\frac{\nu^3}{c^3}\frac{8\pi h}{\exp(h\nu/k_BT)-1}
\end{equation*}

\subsection*{Distribution for Indistinguishable Particle}
The distribution for indistinguishable ideal quantum gas also called the Bose-Einstein distribution. The number particle per unit energy interval is given by 
\begin{equation*}
    n_E=\frac{2\pi}{h^3} \frac{V (2m)^{3/2}\sqrt{E}}{\exp\left[(E-\mu)/k_BT\right]-1}
\end{equation*}

\subsubsection*{Derivation.} Einstein's method has some similarity with Bose's method. He first considers the energy of non-relativistic particle, which is 
\begin{equation*}
    E=\frac{p^2}{2m}
\end{equation*}
This implies
\begin{equation*}
    p=\sqrt{2mE},\quad dp=\frac{m}{p}dE=\frac{m}{\sqrt{2mE}}dE
\end{equation*}
This will be used to determine the volume of $(\mathbf{r},\mathbf{p}+ d\mathbf{p})$ phase space
\begin{multline*}
    d\omega=V \frac{4}{3}\pi\left[(p+dp)^3-p^3\right] = \frac{4}{3}\pi V\left[3p^2\;dp+3p\;d^2p+d^3p\right]\\
    =4\pi Vp^2\;dp =4\pi V \;2mE\;\frac{m}{\sqrt{2mE}}dE=2\pi(2m)^{3/2}\sqrt{E}\;dE
\end{multline*} 
The said volume then divided into cells of $h^3$ volume. Let this number of distinct cells as 
\begin{equation*}
    P_E=\frac{d\omega}{h^3}=\frac{2\pi}{h^3}V(2m)^{3/2}\sqrt{E}\;dE
\end{equation*}
The number of configuration $\mathcal{P}_E$ for distributing $N_E$ indistinguishable particle in $P_E$ boxes is given by 
\begin{equation*}
    \mathcal{P}_E=\frac{(N_E+P_E-1)}{N_E! (P_E-1)!}
\end{equation*}
This gives configuration for particle within $(E,E+dE)$, conversely the total configuration is given by 
\begin{equation*}
    \mathcal{P}=\prod_{E=0}^{\infty}\mathcal{P}_E
\end{equation*}

We then move to determining the entropy $S=k_B \ln \mathcal{P}_{\max}$ by maximizing the logarithm of $\mathcal{P}$ with respect to $N_E$ subjected to the following constraints.
\begin{equation*}
    N=\sum_{E=0}^{\infty}N_E,\quad U=\sum_{E=0}^{\infty}EN_E
\end{equation*}
Using Stirling's approximation on said logarithm, we have 
\begin{align*}
    \ln \mathcal{P}&=\sum_{E}\ln\frac{(N_E+P_E-1)}{N_E! (P_E-1)!}\\
    \ln \mathcal{P}&=\sum_E (N_E+P_E)\ln (N_E+P_E)-N_E\ln(N_E)- P_E \ln (P_E)
\end{align*}
By Lagrange's method
\begin{equation*}
    F=\ln \mathcal{P} +\lambda_1\sum_{E} N_E +\lambda_2\sum_{E}EN_E
\end{equation*}
Setting its derivative to zero
\begin{equation*}
    \ln (N_E+P_E)+1-\ln (N_E)-1+\lambda_1+\lambda_2E=0
\end{equation*}
Taking the exponential and solving for $N_E$
\begin{align*}
    \frac{N_E+P_E}{N_E}&=\exp(-\lambda_1-\lambda_2E)\\
    N_E&=\frac{P_E}{\exp(-\lambda_1-\lambda_2E)-1}
\end{align*}
Substituting this value into the logarithm of $\mathcal{P}$
\begin{multline*}
    \ln \mathcal{P}_{\max}=\sum_E\Bigg[ (N_E+P_E)\left[\ln (P_E)+\ln\left(\frac{\exp(-\lambda_1-\lambda_2E)}{\exp(-\lambda_1-\lambda_2E)-1}\right)\right]\\
    - P_E \ln (P_E)-N_E\left[\ln(P_E)-\ln\left(\exp \{-\lambda_1-\lambda_2E\}-1\right)\right] \Bigg]
\end{multline*}
then 
\begin{multline*}
    \ln \mathcal{P}_{\max}=\sum_E \Bigg[ (N_E+P_E)\ln\left(\frac{\exp(-\lambda_1-\lambda_2E)}{\exp(-\lambda_1-\lambda_2E)-1}\right)\\
    + N_E\ln\left(\exp \{-\lambda_1-\lambda_2E\}-1\right)\Bigg]
\end{multline*}
moreover
\begin{equation*}
    \ln \mathcal{P}_{\max}=\sum_E \Bigg[ N_E \left(-\lambda_1-\lambda_2E\right)
    + P_E\ln\left(\frac{\exp(-\lambda_1-\lambda_2E)}{\exp(-\lambda_1-\lambda_2E)-1}\right) \Bigg]
\end{equation*}
Hence
\begin{equation*}
    S=\sum_E k_B P_E\ln\left(\frac{\exp(-\lambda_1-\lambda_2E)}{\exp(-\lambda_1-\lambda_2E)-1}\right)-k_B(N\lambda_1+U\lambda_2)
\end{equation*}
On using the following relations
\begin{equation*}
    \frac{1}{T}=\frac{\partial S}{\partial U}\bigg|_{V,N},\quad -\frac{\mu}{T}=\frac{\partial S}{\partial N}\bigg|_{U,V}
\end{equation*}
we have 
\begin{align*}
    \frac{1}{T}&=-\lambda_2k_B\\
    \lambda_2&=-\frac{1}{k_BT}
\end{align*}
and
\begin{align*}
    -\frac{\mu}{T}&=\lambda_1k_B\\
    \lambda_1&=\frac{\mu}{k_BT}
\end{align*}
Thus
\begin{equation*}
    N_E=\frac{2\pi}{h^3} \frac{V (2m)^{3/2}\sqrt{E}}{\exp\left[(E-\mu)/k_BT\right]-1} \;dE
\end{equation*}
The quantity $n_E$ is defined as the number of molecules per unit energy interval
\begin{equation*}
    n_E=\frac{N_E}{dE}=\frac{2\pi}{h^3} \frac{V (2m)^{3/2}\sqrt{E}}{\exp\left[(E-\mu)/k_BT\right]-1} \quad\blacksquare
\end{equation*}

\subsubsection*{Distribution of photon.} Suppose we use Einstein method to rederive Planck's law, just like Bose did. In his method, he distributed the number of identical quanta $N_\nu$ within distinct $P_\nu$ cells. The number of configuration in this case is given by 
\begin{equation*}
    \mathcal{P}_\nu=\frac{(N_\nu+P_\nu-1)}{N_\nu! (P_\nu-1)!}
\end{equation*}
The logarithm for total frequency reads
\begin{equation*}
    \ln \mathcal{P}=\sum_\nu (N_\nu+P_\nu)\ln (N_\nu+P_\nu)-N_\nu\ln(N_\nu)- P_\nu \ln (P_\nu)
\end{equation*}

We want to maximize this logarithm constrained by 
\begin{equation*}
    U=h\sum_{\nu=0}^{\infty} \nu N_\nu
\end{equation*}
The constraints on number of quanta simply does not exist since the quanta continuously created and destroyed by the cavity walls. To maximize the said logarithm, we then construct
\begin{equation*}
    F=\ln \mathcal{P}+\lambda_1h\sum_{\nu} \nu N_\nu
\end{equation*}
We then set its derivative to zero 
\begin{equation*}
    \ln(N_\nu+P_\nu) +1 -\ln(N_\nu) -1 +\lambda h\nu=0
\end{equation*}
Hence
\begin{align*}
    \frac{N_\nu+P_\nu}{N_\nu}&\exp(-\lambda_1h\nu)\\
    N_\nu&=\frac{P_\nu}{\exp(-\lambda_1 h\nu)-1}
\end{align*}
Substituting this into the logarithm of $\mathcal{P}$ to obtain
\begin{equation*}
    \ln \mathcal{P}_{\max}=\sum_\nu \Bigg[N_\nu \left(-\lambda_1-\lambda_1 E\right)
    + P_\nu\ln\left(\frac{\exp(-\lambda_1E)}{\exp(-\lambda_1E)-1}\right)\Bigg]
\end{equation*}

The entropy reads
\begin{equation*}
    S=\sum_\nu k_B P_E \ln\left(\frac{\exp(-\lambda_1E)}{\exp(-\lambda_1E)-1}\right)-k_B\lambda_1 U
\end{equation*}
Hence 
\begin{equation*}
    \frac{1}{T}=\frac{\partial S}{\partial U}=-k_B\lambda\implies \lambda_1=\frac{1}{T}
\end{equation*}
Substituting this constant back, we have
\begin{equation*}
    N_\nu=\frac{\nu^3}{c^3}\frac{8\pi h V}{\exp(h\nu/k_BT)-1}\;d\nu,\quad     U_\nu=\frac{\nu^3}{c^3}\frac{8\pi h V}{\exp(h\nu/k_BT)-1}\;d\nu 
\end{equation*}
and
\begin{equation*}
    \mathcal{U}(\nu,T) =\frac{\nu^3}{c^3}\frac{8\pi h}{\exp(h\nu/k_BT)-1}
\end{equation*}
which is the same as Bose's result.
 
\subsection*{Distribution for Distinguishable Particle}
The distribution function is given by 
\begin{equation*}
    n_E=\left(\frac{\sqrt[3]{4} N^{2/3}}{\sqrt{\pi}k_B T}\right)^{3/2} \sqrt{E}\exp\left(-\frac{E}{k_BT}\right)
\end{equation*}

\subsubsection*{Derivation.} The number of way to distribute $N_E$ distinct particle into $P_E$ distinct cells is 
\begin{equation*}
    \mathcal{P}_e=P_E^{N_E}
\end{equation*}
where
\begin{equation*}
    P_E=\frac{2\pi}{h^3}V(2m)^{3/2}\sqrt{E}\;dE
\end{equation*}
Hence the number of distribution for all energy interval is 
\begin{equation*}
    \mathcal{P}_E=\prod_{E=0}^{\infty}P_E^{N_E}
\end{equation*}
The total number of configuration is obtained by multiplying the number of configuration for distinct case $\mathcal{P}_E$ by the number of ways to distribute $N_E$ particle from $N$
\begin{equation*}
    \mathcal{P}=\prod_{E} P_E^{N_E}N!\left(\prod_{E} N_E!\right)^{-1}
\end{equation*}
By applying Stirling's approximation, the logarithm reads
\begin{align*}
    \ln \mathcal{P}&=\sum_E N_E \ln P_E+ N\ln N -N-\sum_E N_E \ln N_E+ N_E\\
    \ln \mathcal{P}&=\sum_E N_E \ln \left(\frac{P_E}{N_E}\right)+ N\ln (N)
\end{align*} 
Then define the following function according to Lagrange's method
\begin{equation*}
    F=\sum_E N_E \ln \left(\frac{P_E}{N_E}\right)+ N\ln (N) +\lambda_1\sum_E N_E +\lambda_2\sum_E EN_E
\end{equation*}
where we have used the $N$ and $U$ constraints. Then set its derivative to zero 
\begin{equation*}
    \ln \left(\frac{P_E}{N_E}\right)-1+\lambda_1+\lambda_2E=0
\end{equation*}
Solving for $N_E$
\begin{align*}
    \frac{P_E }{N_E}&=C_1\exp\left(-\lambda_2E\right)\\
    N_E&=P_E C_2 \exp\left(-\beta E\right)
\end{align*}
Inserting this into $N$ constraint
\begin{equation*}
    N=\sum_E P_E C_2 e^{-\beta E}= \frac{2\pi}{h^3}V(2m)^{3/2} C_2\int_{0}^{\infty} \sqrt{E}e^{-\beta E}\;dE
\end{equation*}

The integral is solved as follows. Consider
\begin{equation*}
    I=\int_{0}^{\infty} \sqrt{x}e^{-ax}\;dx
\end{equation*}
Make change of variable $u=-ax$. We have then $d\alpha=-a\;dx$ and $\sqrt{x}=i\sqrt{\alpha/a}$. The integral then reads
\begin{equation*}
    I=-\frac{i}{a^{3/2}}\int_{0}^{-\infty}\sqrt{\alpha}e^\alpha\;d\alpha
\end{equation*}
By choosing $u=\sqrt{\alpha}$ and $dv=e^\alpha$, consequently we have $du=1/2 \sqrt{\alpha}$ and $v=e^\alpha$. Using method of integral by parts,
\begin{equation*}
    I=-\frac{i}{a^{3/2}}\left[e^\alpha \sqrt{\alpha}\bigg|_{0}^{-\infty} -\int_{0 }^{-\infty }\frac{e^\alpha}{2\sqrt{\alpha}}\;d\alpha\right] =\frac{i}{2a^{3/2}}\int_{0 }^{- \infty }\frac{e^\alpha}{\sqrt{\alpha}}\;d\alpha
\end{equation*}
We again make the change of variable $t^2=\alpha$; which implies $d\alpha=2t \;dt$. Although in this case, the lower limit remains the same, namely zero, the upper limit has two possible value due to quadratic nature of our variable
\begin{equation*}
    t^2=-\infty\implies t=\pm \infty i
\end{equation*}
The value we pick is the negative one; this is due to the nature of logarithm. Recalling the definition of imaginary error function. The integral above can be recast as 
\begin{equation*}
    I= \frac{i}{a^{3/2}}\int_{0 }^{- \infty i}e^{t^2} \;dt=\frac{i\pi}{2a^{3/2}}\erfi(- \infty i)
\end{equation*}
Hence, the result is
\begin{equation*}
    I=\frac{\pi}{2a^{3/2}}
\end{equation*}
a positive value. For positive value $t=-\infty i$, the resulting integral will be negative instead. We want the positive value of the integral since we will be taking the logarithm. Positive argument will ensure the value of our logarithm is real, since it represent the logarithm of number a configuration, which is positive real. 

Now plugging the result of previous integral, we have
\begin{equation*}
    N=\frac{2\pi}{h^3}V(2m)^{3/2}C_2\frac{\sqrt{\pi}}{2\beta^{3/2}}=C_2V\left(\frac{2\pi m}{h^2\beta}\right)^{3/2}
\end{equation*}
or by solving of $C_2$, 
\begin{equation*}
    C_2=\frac{N}{V} \left(\frac{h^2\beta}{2\pi m}\right)^{3/2}
\end{equation*}
And by plugging this into $N_E$, we also have 
\begin{equation*}
    N_E=\frac{N}{V}P_E \exp\left(-\beta E\right) \left(\frac{h^2\beta}{2\pi m}\right)^{3/2}
\end{equation*}

The logarithm of maximum configuration then 
\begin{align*}
    \ln \mathcal{P}_{\max}&=\sum_E N_E\ln\left[ \frac{V}{N}\exp\left(\beta E\right) \left(\frac{2\pi m}{h^2\beta}\right)^{3/2}\right]+N\ln (N)\\
    \ln \mathcal{P}_{\max}&=N\ln(V)+\frac{3}{2}N\ln\left(\frac{2\pi m}{h^2\beta}\right) +\beta U
\end{align*}
As for the entropy
\begin{equation*}
    S=Nk_B\ln(V)+\frac{3}{2}Nk_B\ln\left(\frac{2\pi m}{h^2\beta}\right) +k_B\beta U
\end{equation*}
Using the thermodynamics relationship of 
\begin{equation*}
    \frac{1}{T}=\frac{\partial S}{\partial U}\bigg|_{V,N}= k_B U \implies \beta=\frac{1}{T}
\end{equation*}

Now the number of particle within $(\nu,\nu+d\nu)$ can be evaluated as 
\begin{align*}
    N_E&=\frac{N}{V}\frac{2\pi}{h^3}V(2m)^{3/2}\sqrt{E} \exp\left(-\frac{E}{k_BT}\right) \left(\frac{h^2}{2\pi k_BT m}\right)^{3/2}\;dE\\
    N_E&=\left(\frac{\sqrt[3]{4} N^{2/3}}{\sqrt{\pi}k_B T}\right)^{3/2} \sqrt{E}\exp\left(-\frac{E}{k_BT}\right)\;dE
\end{align*}
and the distribution function as 
\begin{equation*}
    n_E=\left(\frac{\sqrt[3]{4} N^{2/3}}{\sqrt{\pi}k_B T}\right)^{3/2} \sqrt{E}\exp\left(-\frac{E}{k_BT}\right)
\end{equation*}

We can also recover the ideal gas law using the thermodynamics relationship
\begin{equation*}
    \frac{P}{T}=\frac{\partial S}{\partial V}\bigg|_{S,N}=\frac{Nk_B}{V} \implies PV=Nk_BT
\end{equation*}
\end{document}