\documentclass[../../../Main.tex]{subfiles}
\begin{document}
\subsection*{Bose Statistics}
Bose derived Planck's law independent of classical electrodynamics to obtain coefficient $8\pi \nu^2/c^3$. He considered gas of photon with energy of $E=h\nu$ and momentum $p=hv/c$. He assumed that quantum state is described by space and momentum phase space. Next he determined the volume of phase space within interval $(\nu,\nu+d\nu)$, which is
\begin{equation*}
    d\omega=V\;d\mathcal{V}
\end{equation*}
where $V$ is the spatial volume and $d\mathcal{V}$ is the momentum volume of the shell within radii $h(\nu+d\nu)/c,h\nu$. The said shell volume is given by
\begin{multline*}
    d\mathcal{V}=(\mathcal{V}+d\mathcal{V})-\mathcal{V}=\frac{4\pi h^3}{3c^3}\left[(\nu+d\nu)^3-\nu^3\right] =\\ \frac{4\pi h^3}{3c^3}\left[3\nu^2\;d\nu+3\nu\;d^2\nu+d^3\nu\right]=\frac{4\pi h^3}{c^3}\nu^2\;d\nu
\end{multline*}
The resulting volume of phase space $d\omega$ is
\begin{equation*}
    d\omega=4\pi V\left(\frac{h}{c}\right)^3\nu^2\;d\nu
\end{equation*}
This volume then divided by Bose into cells of volume $h^3$. The number $P_\nu$ as phase space cells available to the photon expressed by
\begin{equation*}
    P_\nu=\frac{2}{c^3}d\omega=\frac{8\pi V\nu^2}{c^3}
\end{equation*}
where the factor of 2 comes after taking account the two direction of polarization. 

Next he evaluates the entropy of said system. To do that, he uses the same method as Boltzmann. He defined the cell $\mathcal{P}_v$ as a box which photon are distributed. Let $n_{k\nu}$ defined as the number of boxes that contain $k$ photon of frequency $\nu$. We have then the following constraints.
\begin{equation*}
    P_\nu=\sum_{k=0}^{\infty}n_{k\nu},\quad U_\nu= hv\sum_{k=0}^{\infty}kn_{k\nu}
\end{equation*}
or simply
\begin{equation*}
    P_\nu=\sum_{k=0}^{\infty}n_{k\nu},\quad N_\nu= \sum_{k=0}^{\infty} kn_{k\nu}
\end{equation*}
where $N_\nu$ is the number of photon with frequency $\nu$. The number of configuration is given by 
\begin{equation*}
    \mathcal{P}_\nu=P_\nu!\left(\prod_{k=0}^{\infty}n_{k\nu}\right)^{-1}
\end{equation*}  
All of those equations applies for distinct frequency $\nu$, what we what however distribution over all frequency $(0,\infty)$. The expression for constraints is
\begin{equation*}
    P_\nu=\sum_{k=0}^{\infty}n_{k\nu},\quad U= h\sum_{k=0}v\sum_{k=0}^{\infty}kn_{k\nu}
\end{equation*}
and the expression for the number of configuration is 
\begin{equation*}
    \mathcal{P}=\sum_{\nu=0}^{\infty}\mathcal{P}_\nu
\end{equation*} 
As usual, the equation that we want to maximize not the configuration $\mathcal{P}$ itself, rather its logarithm; which may be written as 
\begin{equation*}
    \ln \mathcal{P}=\sum_{\nu=0}^{\infty}\ln\left[\mathcal{P}_v!\left( \prod_{k=0}^{\infty} n_{k\nu}\right)^{-1}\right]
\end{equation*}
On using Stirling's approximation
\begin{align*}
    \ln \mathcal{P}&=\sum_\nu^\infty\left[\mathcal{P}_\nu\ln \left(\mathcal{P}_\nu\right)-\mathcal{P}_\nu - \sum_{k}^{\infty}\left\{n_{k\nu}\ln \left(n_{k\nu} \right)-n_{k\nu} \right\}\right]\\
    \ln \mathcal{P}&=\sum_\nu^\infty\left[\mathcal{P}_\nu\ln \left(\mathcal{P}_\nu\right) - \sum_{k}^{\infty}n_{k\nu}\ln \left(n_{k\nu} \right)\right]
\end{align*}
Since we want to maximize said logarithm with respect to $n_{k\nu}$, we construct the following function using Lagrange's method.
\begin{equation*}
    F(n_{k\nu})=\sum_\nu \left[\mathcal{P}\ln \mathcal{P}_\nu - \sum_{k} n_{k\nu}\ln n_{k\nu} \right]+ 
    \lambda_1\sum_{k}n_{k\nu}+
    \lambda_2h\sum_{v,k}vkn_{k\nu}
\end{equation*}
Setting its derivative to zero
\begin{equation*}
    \frac{dF}{dn_{k\nu}}=-\ln(n_{k\nu})-1+\lambda_1+\lambda_2hk\nu=0
\end{equation*}
which implies
\begin{equation*}
    n_{k\nu}=\exp\left(1-\lambda_1-\lambda_2hk\nu\right)=C\exp\left(-\lambda_2hk\nu\right)
\end{equation*}
Substituting the result into $\mathcal{P}_\nu$ constraints
\begin{equation*}
    \mathcal{P}_\nu=C\sum_{k } \exp\left(-\lambda_2hk\nu\right) \implies C=\mathcal{P}_\nu\left[\sum_{k } \exp\left(-\lambda_2hk\nu\right)\right]^{-1}
\end{equation*}
The term inside parenthesis is a geometric series with ratio of $\exp(-\lambda_2h\nu)$. The constant then can be simply evaluated into 
\begin{equation*}
    C=\mathcal{P}_\nu\left[\frac{1}{1-\exp(-\lambda_2h\nu)}\right]^{-1} = \mathcal{P}_\nu\left[1-\exp(-\lambda_2h\nu)\right]
\end{equation*}
Hence the $n_{k\nu}$ assumes the form 
\begin{equation*}
    n_{k\nu}=\mathcal{P}_\nu\left[1-\exp(-\lambda_2h\nu)\right]\exp\left(-\lambda_2hk\nu\right)
\end{equation*}
On using this to the expression for logarithm of $\mathcal{P}$, we obtain
\begin{multline*}
    \ln \mathcal{P}_{\max}= \sum_\nu \bigg[ \mathcal{P}_\nu \ln \left(\mathcal{P}_\nu\right) \\
    - \sum_{k} n_{k\nu}\ln \{\mathcal{P}_\nu\left[1-\exp(-\lambda_2h\nu)\right]\exp\left(-\lambda_2hk\nu\right) \} \bigg]
\end{multline*}
Then by definition of Boltzmann entropy, we have
\begin{multline*}
    S_B=k_B \ln \mathcal{P}_{\max}= \sum_\nu k_B \bigg[ \mathcal{P}_\nu \ln \left(\mathcal{P}_\nu\right) \\
    - \sum_{k} n_{k\nu}\ln \{\mathcal{P}_\nu\left[1-\exp(-\lambda_2h\nu)\right]\exp\left(-\lambda_2hk\nu\right) \} \bigg]
\end{multline*}
We then write it as such 
\begin{multline*}
    S_B= \sum_\nu k_B \bigg[ \mathcal{P}_\nu \ln \left(\mathcal{P}_\nu\right) \\
    - \sum_{k} n_{k\nu}\left\{\ln(\mathcal{P}_\nu)+\ln (1-\exp\{-\lambda_2h\nu\})-\lambda_2vkh\right\} \bigg]
\end{multline*}
additionally
\begin{equation*}
    S_B= \sum_\nu k_B \bigg[\lambda_2hv\sum_{k}kn_{k\nu} - \sum_{k} n_{k\nu}\ln (1-\exp\{-\lambda_2h\nu\})\bigg]
\end{equation*}
Recall the $\mathcal{P}_\nu$ and $U_\nu$ constraints
\begin{equation*}
    S_B= k_B \bigg[\lambda_2U- \sum_\nu\mathcal{P}_\nu \ln (1-\exp\{-\lambda_2h\nu\})\bigg]
\end{equation*}
Using the following relationship
\begin{equation*}
    \frac{1}{T}=\frac{\partial S}{\partial U}
\end{equation*}
we have
\begin{equation*}
    \frac{1}{T}=k_B\lambda_2\implies \lambda_2=\frac{1}{k_BT}
\end{equation*}
At last, we have 
\begin{equation*}
    n_{k\nu}=\mathcal{P}_\nu\left[1-\exp\left(-\frac{h\nu}{k_BT}\right)\right]\exp\left(-\frac{hk\nu}{k_BT}\right)
\end{equation*}
and
\begin{equation*}
    S_B= \frac{U}{T}- \sum_\nu k_B\mathcal{P}_\nu \ln \left[1-\exp\left(-\frac{h\nu}{k_BT}\right)\right]
\end{equation*}

Henceforward, we use these result to evaluate the following quantity. We have first the number of quanta frequency $\nu$
\begin{equation*}
    N_\nu=\sum_k kn_{k\nu}=\frac{P_\nu}{\exp(h\nu/k_BT)-1}
\end{equation*}
and the energy within the same frequency
\begin{equation*}
    U_\nu=h\nu N_\nu =\frac{h\nu P_\nu}{\exp(h\nu/k_BT)-1}
\end{equation*}
\end{document}