\documentclass[../../../Main.tex]{subfiles}
\begin{document}
\subsection*{Ensembles}
The maximum entropy principle state that thermal equilibrium is described by such probability distribution of microstates for which the statistical entropy is maximum subject to the constraints on the system.
The constraints of general interest are the ones pertaining to whether the system exchanges energy and/or particles with its environment.
Those constraints define three types of ensembles:
\begin{enumerate}
    \item \textbf{Microcanonical.} Represents a system which is isolated and exchanges neither energy nor particles with the environment. System has fixed $N,\;V,\;E$.
    \item \textbf{Canonical.} Represents a system which exchanges energy but not particles with the environment. System has fixed $N,\;T,\;V$.
    \item \textbf{Grand Canonical.} Represents a system which exchanges energy as well as particles with the environment. System has fixed $\mu,\;V,\;T$.
\end{enumerate}

\subsection*{Partition function}
For canonical ensembles, the partition function is defined as follows
\begin{equation*}
    Z\equiv \sum_i e^{\epsilon_i/kT}
\end{equation*}
where $i$ denote the index for the microstates of the system.
And
\begin{equation*}
    Z\equiv \sum_{s} g_se^{\epsilon_s/kT}
\end{equation*}
where $g_s$ denote the number of microstates with $\epsilon_s$ energy. 
Note the difference of $\epsilon_s$ in the context quantum system and canonical ensembles.
In quantum system, it refers to the $s$-th energy level, while here it refers to the $s$-th microstates.


\subsection*{Probability}
On using the defined partition function, the probability $P_i$ that our system is in microstate $i$ with energy $\epsilon_i$ is
\begin{equation*}
    P_i=\frac{1}{Z}e^{\epsilon_i/kT}
\end{equation*}

\subsubsection*{Derivation.} Probability of the system being in $i$-th state is proportional to the number of ways the reservoir can arrange itself with that its own energy
\begin{equation*}
    P_i\propto \Omega_R(E_T-E_i)
\end{equation*}
since $E_R=E_T-E_i$. By Boltzmann definition of entropy
\begin{equation*}
    \Omega_R(E_T-E_i)=\exp\left[\frac{S_R(E_T-E_i)}{k }\right]
\end{equation*}
Then 
\begin{equation*}
    P_i\propto \exp\left[\frac{S_R(E_T-E_i)}{k }\right]
\end{equation*}
We know that $E_i\ll E_R$, thus $E_i\ll E_T$ and we can expand the reservoir entropy using Taylor expansion
\begin{equation*}
    S_R(E_T-E_i)\approx S_R(E_T)-\frac{\partial S_R(E_T)}{\partial E}E_i
\end{equation*}
Using the relation $\partial S/\partial E=1/T$
\begin{equation*}
    S_R(E_T-E_i)\approx S_R(E_T)-\frac{E_i}{T}
\end{equation*}
So 
\begin{equation*}
    P_i\propto\exp\left[\frac{S_R}{k }\right]\exp\left[-\frac{E_i}{kT}\right]
\end{equation*}
Since $e^{S_R/k}$ is a constant which does not depend on the microstate $i$, we can define it as the normalization constant
\begin{equation*}
    \sum_i P_i=C\sum_i e^{-E_i/kT}=1
\end{equation*}
which can be evaluated as 
\begin{equation*}
    C=\frac{1}{\sum_i e^{-E_i/kT}}=\frac{1}{Z}
\end{equation*}
Therefore
\begin{equation*}
    P_i=\frac{\sum_i e^{-E_i/kT}}{Z}
\end{equation*}
\end{document}