\documentclass[../../../Main.tex]{subfiles}
\begin{document}
\subsection*{Quantum Statistics}
Based on their quantum statistics properties, there are three types of particle, namely 
\begin{enumerate}
    \item \textbf{Classics}. This particle obeys Maxwell-Boltzmann (MB) distribution. It assumes particles are distinguishable and can occupy any energy state without restriction.
    \item \textbf{Boson.} This particle obeys the Bose-Einstein (BE) distribution. It is indistinguishable and can occupy the same quantum state without limit.
    \item \textbf{Fermion.} This particle obeys Fermi-Dirac (FD) distribution. Fermions are indistinguishable and obey the Pauli exclusion principle, meaning no two fermions can occupy the same quantum state.
\end{enumerate}

\subsection*{Discrete System}
\subsubsection*{Microstate.} The number of microstate based on energy level for classical particle is 
\begin{equation*}
    \Omega_\text{MB}=N!\prod_k \frac{g_k^{n_k}}{n_k!}
\end{equation*}
for boson 
\begin{equation*}
    \Omega_\text{BE}=\prod_k\frac{(n_k+g_k-1)!}{n_k!(g_k-1)!}
\end{equation*}
and for fermion
\begin{equation*}
    \Omega_\text{FD}=\prod_k\frac{g_k!}{n_k!(g_k-n_k)!}
\end{equation*}

\subsubsection*{Example.} Consider system with following condition.
\begin{longtable}{c c c}
    \caption*{Table: Arbitrary system}\\
    \hline
    $n$ & $(g_n,\epsilon_n)$\\ 
    \hline\\

    $4$&$(1,3)$\\
    $3$&$(3,2)$\\
    $2$&$(2,1)$\\
    $1$&$(3,0)$\\
\end{longtable}
We have the following macrostate.
\begin{align*}
    M_1&=(2,0,1,1)\\
    M_2&=(1,2,0,1)\\
    M_3&=(1,1,1,2)\\
    M_4&=(0,3,1,0)
\end{align*}
For classical particle, we have also the following number of microstate
\begin{align*}
    \Omega_1&=4!\frac{3^2}{2!}\frac{2^0}{0!}\frac{3^1}{1!}\frac{1^1}{1!}=324\\
    \Omega_2&=144\\
    \Omega_3&=648\\
    \Omega_5&=96
\end{align*}
For boson 
\begin{align*}
    \Omega_1&=\frac{4!1!3!1!}{2!2!1!2!0!}=18\\
    \Omega_2&=9\\
    \Omega_3&=36\\
    \Omega_4&=12
\end{align*}
and for fermion
\begin{align*}
    \Omega_1&=\frac{3!}{2!}\frac{2!}{2!}\frac{3!}{2!}\frac{1!}{1!}=9\\
    \Omega_2&=3\\
    \Omega_3&=18\\
\end{align*}
No $M_4$ for fermion since those three particle cannot have the same energy.
\end{document}