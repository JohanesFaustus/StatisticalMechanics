\documentclass[../../../Main.tex]{subfiles}
\begin{document}
\subsection*{Quantum Statistics}
Based on their quantum statistics properties, there are three types of particle, namely 
\begin{enumerate}
    \item \textbf{Classics}. This particle obeys Maxwell-Boltzmann (MB) distribution. It assumes particles are distinguishable and can occupy any energy state without restriction.
    \item \textbf{Boson.} This particle obeys the Bose-Einstein (BE) distribution. It is indistinguishable and can occupy the same quantum state without limit.
    \item \textbf{Fermion.} This particle obeys Fermi-Dirac (FD) distribution. Fermions are indistinguishable and obey the Pauli exclusion principle, meaning no two fermions can occupy the same quantum state.
\end{enumerate}

\subsection*{Discrete System}
\subsubsection*{Microstate.} The number of microstate based on energy level for classical particle is 
\begin{equation*}
    \Omega_\text{MB}=N!\prod_k \frac{g_k^{n_k}}{n_k!}
\end{equation*}
for boson 
\begin{equation*}
    \Omega_\text{BE}=\prod_k\frac{(n_k+g_k-1)!}{n_k!(g_k-1)!}
\end{equation*}
and for fermion
\begin{equation*}
    \Omega_\text{FD}=\prod_k\frac{g_k!}{n_k!(g_k-n_k)!}
\end{equation*}

\subsubsection*{Example.} Consider system with four particle and five unit energy, where each energy level have the following condition.
\begin{longtable}{c c c}
    \caption*{Table: Random system}\\
    \hline
    $n$ & $(g_n,\epsilon_n)$\\ 
    \hline\\

    $4$&$(1,3)$\\
    $3$&$(3,2)$\\
    $2$&$(2,1)$\\
    $1$&$(3,0)$\\
\end{longtable}
We have the following macrostate.
\begin{align*}
    M_1&=(2,0,1,1)\\
    M_2&=(1,2,0,1)\\
    M_3&=(1,1,2,0)\\
    M_4&=(0,3,1,0)
\end{align*}
For classical particle, we have also the following number of microstate
\begin{align*}
    \Omega_1&=4!\frac{3^2}{2!}\frac{2^0}{0!}\frac{3^1}{1!}\frac{1^1}{1!}=324\\
    \Omega_2&=144\\
    \Omega_3&=648\\
    \Omega_4&=96
\end{align*}
For boson 
\begin{align*}
    \Omega_1&=\frac{4!1!3!1!}{2!2!1!2!0!}=18\\
    \Omega_2&=9\\
    \Omega_3&=36\\
    \Omega_4&=12
\end{align*}
and for fermion
\begin{align*}
    \Omega_1&=\frac{3!}{2!}\frac{2!}{2!}\frac{3!}{2!}\frac{1!}{1!}=9\\
    \Omega_2&=3\\
    \Omega_3&=18\\
\end{align*}
No $M_4$ for fermion since those three particle cannot have the same energy. 

\subsubsection*{Another example.} This time I will do classical particle only. Consider system with $N=E=7$ unit whose each level of energy obey the following condition.
\begin{longtable}{c c c}
    \caption*{Table: Random system}\\
    \hline
    $n$ & $(g_n,\epsilon_n)$\\ 
    \hline\\
    $7$&$(3,6)$\\
    $6$&$(3,5)$\\
    $5$&$(3,4)$\\ 
    $4$&$(3,3)$\\
    $3$&$(3,2)$\\
    $2$&$(3,1)$\\
    $1$&$(3,0)$\\
\end{longtable}
The corresponding macrostates are as follows.
\begin{equation*}
    \begin{array}{l l}
        M_1=(5,0,0,0,0,0,7)& M_7=(2,3,0,1,0,0,0)\\
        M_2=(4,1,0,0,0,1,0)& M_8=(3,0,3,0,0,0,0)\\
        M_3=(4,0,1,0,1,0,0)& M_9=(2,2,2,0,0,0,0)\\
        M_4=(3,2,0,0,1,0,0)& M_{10}=(1,4,1,0,0,0,0)\\
        M_5=(4,0,2,0,0,0,0)&M_{11}=(0,6,0,0,0,0,0)\\
        M_6=(3,1,1,1,0,0,0)&
    \end{array}
\end{equation*}
with the following microstate
\begin{equation*}
    \begin{array}{l l}
        \Omega_1=6\cdot3^6& \Omega_7=60\cdot3^6\\
        \Omega_2=30\cdot3^6& \Omega_8=20\cdot3^6\\
        \Omega_3=30\cdot3^6& \Omega_9=90\cdot3^6\\
        \Omega_4=60\cdot3^6& \Omega_{10}=30\cdot3^6\\
        \Omega_5=15\cdot3^6&\Omega_{11}=3^6\\
        \Omega_6=120\cdot3^6&
    \end{array}
\end{equation*}
\end{document}