\documentclass[../../../Main.tex]{subfiles}
\begin{document}
\subsection*{Quantum Statistics}
Based on their quantum statistics properties, there are three types of particle, namely 
\begin{enumerate}
    \item \textbf{Classics}. This particle obeys Maxwell-Boltzmann (MB) distribution. It assumes particles are distinguishable and can occupy any energy state without restriction.
    \item \textbf{Boson.} This particle obeys the Bose-Einstein (BE) distribution. It is indistinguishable and can occupy the same quantum state without limit.
    \item \textbf{Fermion.} This particle obeys Fermi-Dirac (FD) distribution. Fermions are indistinguishable and obey the Pauli exclusion principle, meaning no two fermions can occupy the same quantum state.
\end{enumerate}

\subsection*{Discrete System}
\subsubsection*{Microstate.} The number of microstate based on energy level for classical particle is 
\begin{equation*}
    \Omega_\text{MB}=N!\prod_k \frac{g_k^{n_k}}{n_k!}
\end{equation*}
for boson 
\begin{equation*}
    \Omega_\text{BE}=\prod_k\frac{(n_k+g_k-1)!}{n_k!(g_k-1)!}
\end{equation*}
and for fermion
\begin{equation*}
    \Omega_\text{FD}=\prod_k\frac{g_k!}{n_k!(g_k-n_k)!}
\end{equation*}

\subsubsection*{Example.} Consider system with four particles and five unit energy, where each energy level have the following condition.
\begin{table}[h]
    \centering
    \begin{tabular}{cc}
    \toprule
    $n$ & $(g_n,\epsilon_n)$\\ 
    \midrule
    $4$&$(1,3)$\\
    $3$&$(3,2)$\\
    $2$&$(2,1)$\\
    $1$&$(3,0)$\\
    \bottomrule
    \end{tabular}
\end{table}
    
We have the following macrostate.
\begin{align*}
    M_1&=(2,0,1,1)\\
    M_2&=(1,2,0,1)\\
    M_3&=(1,1,2,0)\\
    M_4&=(0,3,1,0)
\end{align*}
For classical particle, we have also the following number of microstate
\begin{align*}
    \Omega_1&=4!\frac{3^2}{2!}\frac{2^0}{0!}\frac{3^1}{1!}\frac{1^1}{1!}=324\\
    \Omega_2&=144\\
    \Omega_3&=648\\
    \Omega_4&=96
\end{align*}
For boson 
\begin{align*}
    \Omega_1&=\frac{4!1!3!1!}{2!2!1!2!0!}=18\\
    \Omega_2&=9\\
    \Omega_3&=36\\
    \Omega_4&=12
\end{align*}
and for fermion
\begin{align*}
    \Omega_1&=\frac{3!}{2!}\frac{2!}{2!}\frac{3!}{2!}\frac{1!}{1!}=9\\
    \Omega_2&=3\\
    \Omega_3&=18\\
\end{align*}
No $M_4$ for fermion since those three particle cannot have the same energy. 

\subsubsection*{Another example.} This time I will do classical particle only. Consider system with $N=E=7$ unit whose each level of energy obey the following condition.

\begin{table}[h]
    \centering
    \begin{tabular}{cc}
        \toprule
        $n$ & $(g_n,\epsilon_n)$\\ 
        \midrule
        $7$&$(3,6)$\\
        $6$&$(3,5)$\\
        $5$&$(3,4)$\\ 
        $4$&$(3,3)$\\
        $3$&$(3,2)$\\
        $2$&$(3,1)$\\
        $1$&$(3,0)$\\
        \bottomrule
    \end{tabular}
\end{table}
The corresponding macrostates are as follows.
\begin{equation*}
    \begin{array}{l l}
        M_1=(5,0,0,0,0,0,7)& M_7=(2,3,0,1,0,0,0)\\
        M_2=(4,1,0,0,0,1,0)& M_8=(3,0,3,0,0,0,0)\\
        M_3=(4,0,1,0,1,0,0)& M_9=(2,2,2,0,0,0,0)\\
        M_4=(3,2,0,0,1,0,0)& M_{10}=(1,4,1,0,0,0,0)\\
        M_5=(4,0,2,0,0,0,0)&M_{11}=(0,6,0,0,0,0,0)\\
        M_6=(3,1,1,1,0,0,0)&
    \end{array}
\end{equation*}
with the following microstate
\begin{equation*}
    \begin{array}{l l}
        \Omega_1=6\cdot3^6& \Omega_7=60\cdot3^6\\
        \Omega_2=30\cdot3^6& \Omega_8=20\cdot3^6\\
        \Omega_3=30\cdot3^6& \Omega_9=90\cdot3^6\\
        \Omega_4=60\cdot3^6& \Omega_{10}=30\cdot3^6\\
        \Omega_5=15\cdot3^6&\Omega_{11}=3^6\\
        \Omega_6=120\cdot3^6&
    \end{array}
\end{equation*}

\subsubsection*{Buku Pak Rouf, soal nomor 1.} Consider 
\begin{quotation}
    Tujuh buah partikel klasik didistribusikan pada lima tingkat energi dengan $\epsilon_s = (s-1) $ eV dan $g_s =3$, dengan indeks $s$ adalah nomor tingkat energi. Jika energi total partikel adalah 8 eV, tentukanlah: Jumlah keadaan makro $M$ dan spesifikasinya; Jumlah keadaan mikro untuk tiap $M$; Bobot konfigurasi total $W_T$; Bilangan penempatan rata-rata $\braket{n_s}$; dan Peluang untuk mendapatkan partikel dengan energi 2 eV pada pengambilan pertama
\end{quotation}

Keadaan sistem dapat digambarkan sebagai berikut.

\begin{table}[h]
    \centering
    \caption*{Tabel: Keadaan sistem}
    \begin{tabular}{ccc}
    \toprule
    $s$ & $g_s$& $\epsilon_s= (s-1)$ eV\\ 
    \midrule
    $5$& $3 $ & $4 $ \\ 
    $4$& $3 $ & $3 $ \\
    $3$& $3 $ & $2 $ \\
    $2$& $3 $ & $1 $\\
    $1$& $3 $ & $0 $\\
    \bottomrule
    \end{tabular}
\end{table}
Sistem memiliki 12 keadaan makro dengan spesifikasi sebagai berikut.
\begin{equation*}
    \begin{array}{l r}
        M_1=(5,0,0,0,2)& M_7=(3,1,2,1,0)\\
        M_2=(4,1,0,1,1)& M_8=(2,3,1,1,0)\\
        M_3=(4,0,2,0,1)& M_9=(3,0,4,0,0)\\
        M_4=(2,4,0,0,1)& M_{10}=(2,2,3,0,0)\\
        M_5=(4,0,1,2,0)&M_{11}=(1,4,2,0,0)\\
        M_6=(3,2,0,2,0)&M_{12}=(0,6,1,0,0)\\
    \end{array}
\end{equation*}
Jumlah keadaan mikro untuk setiap keadaan makro adalah sebagai berikut.

\begin{align*}
    W_1&=7!\five\zero\zero\zero\two=45\;927\\
    W_2&=7!\four\one\zero\one\one=459\;720\\
    W_3&=7!\four\zero\two\zero\one=229\;635\\
    W_4&=7!\two\four\zero\zero\one=229\;635\\
    W_5&=7!\four\zero\one\two\zero=229\;635\\
    W_6&=7!\three\two\zero\two\zero=102\;160\\
    W_7&= 7!\three\one\two\one\zero=302\;180\\
    W_8&=7!\two\three\one\one\zero=918\;540 \\
    W_9&=7!\three\zero\four\zero\zero =76\;545\\
    W_{10}&=7!\two\two\three\zero\zero =459\;720\\
    W_{11}&=7!\one\four\two\zero\zero =229\;635\\
    W_{12}&=7!\zero\six\one\zero\zero =15\;309
\end{align*}

Untuk menghitung bilangan penempatan rata, diperlukan jumlah keadaan mikro total:
\begin{multline*}
    W_T=\sum_{k=1}^{12} W_k= 45\;927+ 459\;270+ 229\;635 \\
    + 229\;635 + 229\;635 + 102\;060 + 306\;180+ 918\;540\\
    + 76\;545+459\;270+ 229\;635+ 15\;309= 3\;301\;641
\end{multline*}
Jumlah partkel rata-rata untuk tingkat energi pertama $s=1$ adalah 
\begin{multline*}
    \braket{n_1}=\frac{\sum_k n_1W_k}{W_T}=\frac{1}{3\;301\;641}( 5\cdot45927+4\cdot459270+4\cdot229635\cdot\\
    +2\cdot229635\cdot+4\cdot229635\cdot+3\cdot102060+3\cdot306180+2\cdot918540\\+3\cdot76545
    +2\cdot459270+1\cdot229635+0\cdot15309)\approx2.666
\end{multline*}
Untuk tingkat energi kedua
\begin{multline*}
    \braket{n_2}=\frac{\sum_k n_2W_k}{W_T}=\frac{1}{3\;301\;641}(0\cdot45927 + 1\cdot459270 + 0\cdot229635 \\
    + 4\cdot229635 + 0\cdot229635 + 2\cdot102060 + 1\cdot306180 + 3\cdot918540\\ + 0\cdot76545
     + 2\cdot459270 + 4\cdot229635 + 6\cdot15309)\approx 1.991
\end{multline*}
Untuk tingkat energi ketiga
\begin{multline*}
    \braket{n_3}=\frac{\sum_k n_3W_k}{W_T}=\frac{1}{3\;301\;641} (0\cdot45927 + 0\cdot459270 + 2\cdot229635 \\
    + 0\cdot229635 + 1\cdot229635 + 0\cdot102060 + 2\cdot306180 + 1\cdot918540 \\
    + 4\cdot76545 + 3\cdot459270 + 2\cdot229635 + 1\cdot15309)\approx1.326
\end{multline*}
Untuk tingkat energi keempat
\begin{multline*}
    \braket{n_4}=\frac{\sum_k n_4W_k}{W_T}=\frac{1}{3\;301\;641}  (0\cdot45927 + 1\cdot459270 + 0\cdot229635 \\
    + 0\cdot229635 + 2\cdot229635 + 2\cdot102060 + 1\cdot306180 + 1\cdot918540 \\
    + 0\cdot76545 + 0\cdot459270 + 0\cdot229635 + 0\cdot15309)\approx 0.7109
\end{multline*}
Dan untuk tingkat kelima
\begin{multline*}
    \braket{n_4}=\frac{\sum_k n_4W_k}{W_T}=\frac{1}{3\;301\;641}  (2\cdot45927 + 1\cdot459270 + 1\cdot229635 \\
    + 1\cdot229635 + 0\cdot229635 + 0\cdot102060 + 0\cdot306180 + 0\cdot918540 \\
    + 0\cdot76545 + 0\cdot459270 + 0\cdot229635 + 0\cdot15309)\approx 0.3060
\end{multline*}
Sebagai pembuktian,
\begin{equation*}
    N=\sum_s^5n_s=2.666+ 1.991+1.326+ 0.7109+0.3060=7
\end{equation*}
sesuai keadaan sistem.

Kemungkinan mendapatkan partikel dengan energi 2 eV, atau dengan tingkat energi ketiga, adalah
\begin{equation*}
    P(\epsilon_3)=\frac{\braket{n_3}}{N}=\frac{1,326}{7}\approx19\%
\end{equation*}
\end{document}