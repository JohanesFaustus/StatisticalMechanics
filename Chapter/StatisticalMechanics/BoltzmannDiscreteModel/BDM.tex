\documentclass[../../../Main.tex]{subfiles}
\begin{document}
\subsection*{Discrete Energy Levels}
In this model, Boltzmann postulates in a gas of $N$ particle, that each particle has discretely spaced value of energy kinetic $\epsilon P$. The permutation of configuration $E_k|N\equiv E_1,\dots E_N$ denote distinct configuration. State of system is then defined as set $n_k\equiv n_0,\cdot, n_P$ where $n_k$ is the number of molecule having $k\epsilon$ energy level.

\begin{table*}[ht]
    \begin{center}
\caption*{Table: system with two possible energy level $(0, \epsilon)$}
\begin{tabular}{c || c}
    \hline\hline
    $\begin{array}{c}\text{State} \\n_{k|P}=(n_0,n_1)\end{array}$  &  $\begin{array}{c}\text{Configuration} \\ E_k|N=(E_1 , E_2 , E_3 )\end{array}$ \\
    \hline\hline
    $(3, 0)$&$(0, 0, 0)$\\
    $(2, 1)$&$(\epsilon, 0, 0), (0, \epsilon, 0), (0, 0, \epsilon)$\\
    $(1, 2)$&$(\epsilon, \epsilon, 0), (\epsilon, 0, \epsilon), (0, \epsilon, \epsilon)$\\
    $(0, 3)$&$(\epsilon, \epsilon, \epsilon)$\\
\end{tabular}
    \end{center}
\end{table*}

Each configuration must also obey the following restriction.
\begin{equation*}
    \sum_{k=0}^{P}=N,\quad U=\epsilon\sum_{k=0}^{P} kn_k
\end{equation*}
or simply
\begin{equation*}
    \sum_{k=0}^{P}=N,\quad \sum_{k=0}^{P} kn_k=L
\end{equation*}
The first restriction says that each configuration is in such way that the sum of each element $n_k$ is the total number of particle $N$, while the second restriction rule the total energy $U$ of the system.

To determine the total configuration of a specific configuration $n_{k|P}$, we use
\begin{equation*}
    D(N,P,n_{k|P})=\frac{N!}{\displaystyle\prod_{i=0}^{P}n_i!}
\end{equation*}
To find the total configuration of a system, we're then summing all possible state that can be achieved by the system in question. After that, we obtain
\begin{equation*}
    D_T(N,P)=(P+1)^N
\end{equation*}
For a special case when $L\leq P$, the equation above turns into
\begin{equation*}
\mathcal{D}(N,L)=\frac{1}{L!}\frac{(N+L-1)!}{(N-1)!}
\end{equation*}

\begin{longtable}{c||c}
    \caption*{Table: State and configuration a system with $N=P=7$}, and $L\leq P$\\
    \hline\hline
    $\begin{array}{c}\text{State} \\n_{k|P}=(n_0,n_1,n_2, n_3, n_4, n_5, n_6, n_7)\end{array}$&$\begin{array}{c}\text{Number of configuration}\\  D(N,P,n_{k|P})\end{array}$\\ 
    \hline\hline\\

    $(6, 0, 0, 0, 0, 0, 0, 1)$&$\dfrac{7!}{6!0!0!0!0!0!0!1!} =7 $\\\\
    $(5, 1, 0, 0, 0, 0, 1, 0)$&$\dfrac{7!}{5!1!0!0!0!0!1!0!} =42 $\\\\
    $(5, 0, 1, 0, 0, 1, 0, 0)$&$\dfrac{7!}{5!0!1!0!0!1!0!0!} =42$\\\\
    $(5, 0, 0, 1, 1, 0, 0, 0)$&$\dfrac{7!}{5!0!0!1!1!0!0!0!} =42 $\\\\
    $(4, 2, 0, 0, 0, 1, 0, 0)$&$\dfrac{7!}{4!2!0!0!0!1!0!0!} =105$\\\\
    $(4, 1, 1, 0, 1, 0, 0, 0)$&$\dfrac{7!}{4!1!1!0!1!0!0!0!} =210 $\\\\
    $(4, 0, 2, 1, 0, 0, 0, 0)$&$\dfrac{7!}{4!0!2!1!0!0!0!0!} =105 $\\\\
    $(4, 1, 0, 2, 0, 0, 0, 0)$&$\dfrac{7!}{4!1!0!2!0!0!0!0!} =105 $\\\\
    $(3, 3, 0, 0, 1, 0, 0, 0)$&$\dfrac{7!}{3!3!0!0!1!0!0!0!} =140 $\\\\
    $(3, 2, 1, 1, 0, 0, 0, 0)$&$\dfrac{7!}{ 3!2!1!1!0!0!0!0!} =420 $\\\\
    $(3, 1, 3, 0, 0, 0, 0, 0)$&$\dfrac{7!}{3!1!3!0!0!0!0!0!} =140 $\\\\
    $(2, 4, 0, 1, 0, 0, 0, 0)$&$\dfrac{7!}{2!4!0!1!0!0!0!0!} =105 $\\\\
    $(2, 3, 2, 0, 0, 0, 0, 0)$&$\dfrac{7!}{2!3!2!0!0!0!0!0!} =210 $\\\\
    $(1, 5, 1, 0, 0, 0, 0, 0)$&$\dfrac{7!}{1!5!1!0!0!0!0!0!} =42 $\\\\
    $(0, 7, 0, 0, 0, 0, 0, 0)$&$\dfrac{7!}{0!7!0!0!0!0!0!0!} =1 $\\\\
\end{longtable}

\subsection*{ Real\textsuperscript{\texttrademark} System}
Boltzmann postulates that thermal equilibrium correspond to state with the largest number of configuration. The previous example with $N=7$ we know that the state in question is $n_{k|P}=(3, 2, 1, 1, 0, 0, 0, 0)$. In real system with large $N$, it is impossible to determine the equilibrium state using method above.

By Stirling’s formula, the logarithm of $D(N , P, n_{k|P} )$ is expressed as
\begin{equation*}
    \ln\left[D(N , P, n_{k|P} )\right]=N\ln N-N-\sum_{k=0}^{P}\left(n_k\ln n_k-n_k\right)
\end{equation*}

Using equation above, Boltzmann then derive the logarithm of the largest number of configuration, which of course correspond to equilibrium state. The logarithm in question expressed as 
\begin{equation*}
    \ln(\mathcal{D}_\text{max})= (N+L)\ln (N+L) - L\ln (L)-N\ln (N)
\end{equation*}
The number of particle $n_k$ inside configuration above is 
\begin{equation*}
    n_k=N(1-x)x^k
\end{equation*}
where $x=L/(L+N)$. The expression $n_k$ above maximize the $D$. If the average kinetic energy $u=U/N=L\epsilon/N$ is much bigger separation $\epsilon$, $n_k$ acn be approximated as 
\begin{equation*}
    n_k=\frac{N_\epsilon}{u+\epsilon}\left(1+\frac{\epsilon}{u}\right)^{-k}\approx\frac{N\epsilon}{u}e^{-k\epsilon/u}
\end{equation*}

\subsubsection*{Derivation.} To find the desired maximum function, we use Lagrange's multiplier method. We will maximize $D(N , P, n_{k|P})$ for $P\rightarrow\infty$
\begin{equation*}
    F(n_k)=\ln\left[D(N , P, n_{k|P} )\right]- \sum_{k=0}^{P}(\alpha + k\gamma )n_k
\end{equation*}
with respect to $n_k$. Invoking Stirling's formula for$\ln\left[D(N , P, n_{k|P} )\right]$, we have
\begin{equation*}
    F(n_k)=N\ln N-N-\sum_{k=0}^{P}(\ln n_k -1 +\alpha + k\gamma )n_k
\end{equation*}
We will now begin the maximization by
\begin{equation*}
    \frac{\partial F}{\partial n_k}=0\implies\begin{array}{r l}
        \dfrac{\partial }{\partial n_k}\left(n_k\ln n_k\right)-1+ \alpha +k\gamma&=0\\
        \ln n_k+\alpha +k\gamma&=0
    \end{array}
\end{equation*}
Solving for $n_k$
\begin{equation*}
    n=e^{-\alpha-k\gamma}=\left(e^{-\alpha}\right)\left(e^{-\gamma}\right)^k
\end{equation*}
for convenience’s sake, we use 
\begin{equation*}
    n_k=Ax^k\quad \text{with}\quad A=e^{-\alpha}\land x=e^{-\gamma}
\end{equation*}

Using this result for $n_k$, the first restriction $R_\Romannum{1}$ can be written as
\begin{equation*}
    \sum_{k=0}^{P}n_k=N\implies A\sum_{k=0}^{P}x^k=N
\end{equation*}
the series in the equation above is a simple geometric series
\begin{equation*}
    \sum_{k=0}^{P}x^k=1+x+x^2+\cdots x^P=\sum_{k=1}^{P+1}x^{k-1}
\end{equation*}
which can be evaluated as 
\begin{equation*}
    A\frac{1-x^{P+1}}{1-x}=N
\end{equation*}
Hence 
\begin{equation*}
    A=N\frac{1-x}{1-x^{P+1}}
\end{equation*}

Whereas the second restriction reads
\begin{equation*}
    \sum_{k=0}^{P}kn_k=L\implies A\sum_{k=0}^{P}kx^k=L
\end{equation*}
This series is more complicated than before, however it still might be evaluated into 
\begin{equation*}
    L=Ax\frac{\left\{\left[P\left(x-1\right)-1\right]x^P+1\right\}}{(x-1)^2}
\end{equation*}  
where we have invoked WolframAlpha to evaluate the said series. For a real\textsuperscript{\texttrademark} system, we have $P\rightarrow\infty $, therefore those two expression turn into
\begin{equation*}
    A=\lim_{P\rightarrow\infty}N\frac{1-x}{1-\exp (-\gamma P-\gamma)}=N(1-x)
\end{equation*}  
and
\begin{equation*} 
    L=\frac{Ax}{(x-1)^2}\lim_{P\rightarrow\infty}\left[(P(x-1)e^{-\gamma P}+1)+1\right]=-\frac{Nx(x-1)}{(x-1)^2}=\frac{Nx}{1-x}
\end{equation*}
Rearranging the equation above 
\begin{equation*}
    x=\frac{L}{L+N}
\end{equation*}

Since have found the expression for $A$, $L$, and $x$, we can then write the complete expression for $n_k$
\begin{equation*}
    n_k=\frac{N(1-x)}{1-x^{P+1}}x^k
\end{equation*}
applying the condition for real system, we have $n_k$ which maximize the configuration
\begin{equation*}
    n_k=N(1-x)x^k\lim_{P\rightarrow\infty}\frac{1}{1-\exp\left[-\gamma (P+1)\right]}=N(1-x)x^k
\end{equation*}

Substituting $n_k$ we just obtained inside the expression of $\ln D$, we get 
\begin{equation*}
    \ln(\mathcal{D}_\text{max})=-N\left(\ln (1-x)+\frac{x}{1-x}\ln (x)\right)
\end{equation*}
expressing the equation above in terms of $L$ and $N$ to get 
\begin{equation*}
    \ln(\mathcal{D}_\text{max})= (N+L)\ln (N+L) - L\ln (L)-N\ln (N)
\end{equation*}

\subsection*{Result}
We shall now discuss the result of Boltzmann derivation. The case in this discussion will be the same as previously, which is $N = P = 7 , L \leq P$. Here we will compare those three result: 
\begin{enumerate}
    \item \textbf{Small system}. By this consideration, we know the equilibrium state represented by the following state \begin{equation*}
        n_{k|P}=(3, 2, 1, 1, 0, 0, 0, 0)
    \end{equation*} which has 420 number of configuration.
    \item \textbf{Large system}. Tools we used in this consideration are Stirling's approximation and Lagrange multiplier. We obtain the formula for number of particle $n_k$ with $k\epsilon$ energy \begin{equation*}
        n_k=\frac{N(1-x)}{1-x^{P+1}}x^k
    \end{equation*}
    where $x$ is obtained by solving the following equation
    \begin{equation*}
        (N P - L)x^{P+2} - (NP + N - L)x^{P+1} + (L + N ) = 0.
    \end{equation*}
    Boltzmann solved the equation numerically and obtained $x=0.5078125$.
    \item \textbf{Large system with large \emph{P} approximation.} The equation for $n_k$ in this consideration is written as \begin{equation*}
        n_k= N (1 - x)x^k 
    \end{equation*}
\end{enumerate}
\begin{longtable}{c||c||c||c}
    \caption*{Table: The result of those three considerations. Quite accurate except few numbers.}\\
    \hline\hline
    $k$&$\begin{array}{c}
        n_k\\
        \text{for small system}
    \end{array}$ &$\begin{array}{c}
        n_k\\
        \text{for large system}
    \end{array}$ &$\begin{array}{c}
        \text{Same but with large}\\
        P\text{ approximation}
    \end{array}$\\ 
    \hline\hline
    0 &3&3.4535&3.5 \\
    1 &2&1.7574&1.75 \\
    2 &1&0.8943&0.875 \\
    3 &1&0.4551&0.4375 \\
    4 &0&0.2316&0.2187 \\
    5 &0&0.1178&0.10937 \\
    6 &0&0.0599&0.05468 \\
    7 &0&0.0304&0.02734 \\
\end{longtable}
\end{document}