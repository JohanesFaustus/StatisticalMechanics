\documentclass[../../../Main.tex]{subfiles}
\begin{document}
Both entropy and internal energy are a function of volume and number of particle, which are extensive parameters. In experiment, however, we only controls intensive parameters, such as temperature and pressure. Thermodynamics potentials provide a way to describe system using intensive parameters.

\subsection{Entropy and internal Energy Relation}
I already derived these somewhere, I don't remember
\begin{equation*}
    \frac{\partial U}{\partial S}\bigg|_{V, N_{i} }=T,\quad
    -\frac{\partial U}{\partial V}\bigg|_{S, N_{i} }=P,\quad
    \frac{\partial U}{\partial N}\bigg|_{S, V, N_{i\neq j} }=\mu_j;
\end{equation*}
and 
\begin{equation*}
    \frac{\partial S}{\partial U}\bigg|_{V, N_i}=\frac{1}{T},\quad
    \frac{\partial S}{\partial V}\bigg|_{U, N_i}=\frac{P}{T},\quad
    \frac{\partial S}{\partial N_j}\bigg|_{U, N_{i\neq j}}=-\frac{P}{T}\quad
\end{equation*}

\subsection{Helmholtz Potential}
By performing Legendre transform of $U(S,V,N)$ with respect to $S$, we get
\begin{align*}
    \mathcal{L}_S[U(S,V,N)]&=U\left[S\left(\frac{\partial U}{\partial S}\right), V,N\right]-\frac{\partial U}{\partial S}S\left(\frac{\partial U}{\partial S}\right)\\
    \mathcal{L}_S[U(S,V,N)]&=U[S(T),V,N]-TS(T)
\end{align*}
We define the resulting function as Helmholtz potential or Free energy
\begin{equation*}
    F(T,V,N)=U[S,V,N]-TS
\end{equation*}

\subsubsection{Partial derivative relation.} The differential of $F$ read as
\begin{align*}
    dF&=dU-T\;dS-S\;dT=T\;dS-P\;dV+\mu \;dN-T\;dS-S\;dT\\
    dF&=-S\;dT - P\; dV + \mu \;dN
\end{align*}
Therefore
\begin{equation*}
    S=-\frac{\partial F}{\partial T}\bigg|_{V,N},\quad P=-\frac{\partial F}{\partial V}\bigg|_{T,N},\quad \mu=\frac{\partial F}{\partial N}\bigg|_{T,V}
\end{equation*}

\subsubsection{Physical meaning.} Consider system undergoing thermodynamics transform from state $A$ to $B$. Suppose that the temperature $T$, volume $V$, and number of particle $N$ is held constant during the transformation; according to the second law,
\begin{equation*}
    \int_\text{irr}\frac{\dbar Q}{T}\leq\int_\text{rev}\frac{\dbar Q}{T} \implies \frac{\dbar Q}{T}\leq d S\implies d Q\leq T\;d S
\end{equation*}
\subsubsection{Partial derivative relation.} The differential of $F$ is then 
\begin{equation*}
    dF=dU-T\;dS\implies dF\leq dU-\dbar Q
\end{equation*}
Using the first law and the result from the second law, we have
\begin{equation*}
    dF\leq0\quad\text{for }dT=dV=dN=0
\end{equation*} 
Therefore, Helmholtz potential never increases in thermodynamics transformation which is performed under isothermal, isovolume, and iso-number-of-particle conditions.

\subsection{Gibbs Potential}
Gibbs potential is obtained by performed Legendre transformation to $U(S,V,N)$ with respect $S$ and $V$. Consider
\begin{align*}
    \mathcal{L}_{S,V}[U(S,V,N)]&=U\left[S\left(\frac{\partial U}{\partial S}\right), V\left(\frac{\partial U}{\partial V}\right),N\right]\\
    &-\frac{\partial U}{\partial S}S\left(\frac{\partial U}{\partial S}\right) - \frac{\partial U}{\partial V}V\left(\frac{\partial U}{\partial V}\right)\\
    \mathcal{L}_S[U(S,V,N)]&=U[S(T),V(P),N]-TS(T)+PV(P)
\end{align*}
Hence
\begin{equation*}
    G(T,P,N)=U(S,V,N)-TS+PV
\end{equation*}
By invoking the Euler equation in terms of internal energy, we can also write Gibbs potential in terms of chemical potential 
\begin{align*}
    G(T,P,N)=TS-PV+\mu N-TS+PV=\mu N
\end{align*}
\subsubsection{Partial derivative relation.} Now, consider the total differential of $G$
\begin{align*}
    dG&=dU-T\;dS-S\;dT+P\;dV+V\;dP\\
    &=T\;dS-P\;dV+\mu \;dN-T\;dS-S\;dT+P\;dV+V\;dP\\
    dG&=-S\;dT+V\;dT+\mu\;dN
\end{align*}
It follows that 
\begin{equation*}
    S=-\frac{\partial G}{\partial T}\bigg|_{P,N},\quad V=\frac{\partial G}{\partial P}\bigg|_{T,N},\quad \mu=\frac{\partial G}{\partial N}\bigg|_{T,P}
\end{equation*}

\subsubsection{Physical meaning.} We again consider the same system under transformation. The transformation, however, performed under the same temperature $T$, pressure $P$, and number of particle $N$. We write the differential of $G$ as 
\begin{equation*}
    dG=dU-T\;dS+P\;dV=\dbar Q-T\;dS
\end{equation*}
Using the result we obtained earlier from the second law
\begin{equation*}
    dG \leq0, \quad \text{for }\quad dT=dP=dN=0
\end{equation*}
We have proofed that in isothermal, isobaric, and iso-number-of-particle, Gibbs potential never increases.
\subsection{Enthalpy}
Suppose we perform Legendre transformation of $U(x,y,z)$ with respect to $V$, we write 
\begin{align*}
    \mathcal{L}_V\left[U(S,V,N)\right]&= U\left[S,V\left(\frac{\partial U}{\partial V}\right),N\right]- \frac{\partial U}{\partial V}V\left(\frac{\partial U}{\partial V}\right) \\
    \mathcal{L}_V\left[U(S,V,N)\right]&=U\left[S,V(P),N\right]+PV(P)
\end{align*}
This function is defined as enthalpy
\begin{equation*}
    H(S,P,N)=U(S,V,N)+PV
\end{equation*}
\subsubsection{Partial derivative relation.} We write the differential of $H$ as
\begin{align*}
    dH&=dU+P\;dV+V\;dP=T\;dS-P\;dV+\mu\;dN+P\;dV+V\;dP\\
    dH&=T\;dS+V\;dP+\mu\;dN
\end{align*}
Hence
\begin{equation*}
    T=\frac{\partial H}{\partial S}\bigg|_{P,N}, \quad V=\frac{\partial H}{\partial P}\bigg|_{S,N},\quad \mu=\frac{\partial H}{\partial N}\bigg|_{S,P}
\end{equation*}

\subsubsection{Physical meaning.} We can also write the differential of $H$ as 
\begin{align*}
    dH&=dU+P\;dV+V\;dP=\dbar Q-P\;dV+\mu\;dN+P\;dV+V\;dP\\
    dH&=\dbar Q+V\;dP+\mu\;dN
\end{align*}
This equation show that in an isobaric and iso-number-of-particle process, change in enthalpy is the amount of heat absorbed. Enthalpy is often called the heat function.

Now we consider again the same thermodynamics transformation as before with the following modification of constrains: entropy $S$, pressure $P$, and number of particle $N$ are made to be constant. By the last result, we write the differential of $H$ as 
\begin{equation*}
    dH=\dbar Q+V\;dP+\mu\;dN\implies dH\leq T\;dS+V\;dP+\mu\;dN
\end{equation*}
Applying the transformation constrains, we get 
\begin{equation*}
    dH\leq 0 \quad\text{for}\quad dS=dP=dN=0
\end{equation*}
In other words, enthalpy can never increase in isentropic, isobaric, and iso-number-particle.

\subsection{Grand Potential}
Grand potential is obtained by performing Legendre transform of $U(S,V,N)$ with respect to $S$ and $N$
\begin{align*}
    \mathcal{L}_{S,N}\left[U(S,V,N)\right]&= U\left[S\left(\frac{\partial U}{\partial S}\right),V,N\left(\frac{\partial U}{\partial N}\right)\right] - \frac{\partial U}{\partial S}S\left(\frac{\partial U}{\partial S}\right)\\
    &- \frac{\partial U}{\partial N}N\left(\frac{\partial U}{\partial N}\right) \\
    \mathcal{L}_{S,N}\left[U(S,V,N)\right]&=U\left[S(T),V,N(\mu)\right]- TS(T)-\mu N(\mu)
\end{align*}
Hence
\begin{equation*}
    \Omega(T,V,\mu)=U(S,V,N)-TS-\mu N
\end{equation*}
Due to Euler's equation, we can rewrite the grand potential it terms of pressure and volume
\begin{equation*}
    \Omega(T,V,\mu)=TS-PV+\mu N-TS-\mu N=-PV
\end{equation*}

\subsubsection{Partial differential relation.} We can write the total differential of $\Omega$ as 
\begin{align*}
    d\Omega&= dU-T\;dS- S\;dT-\mu \;dN-N\;d\mu\\
    &=T\;dS-P\;dV+\mu\;dN-T\;dS- S\;dT-\mu \;dN-N\;d\mu\\
    d\Omega&=-P\;dV-S\;dT-N\;d\mu
\end{align*}
And Hence 
\begin{equation*}
    S=-\frac{\partial \Omega}{\partial T}\bigg|_{V,\mu},\quad P=-\frac{\partial \Omega}{\partial V}\bigg|_{T,\mu},\quad N=-\frac{\partial \Omega}{\partial \mu}\bigg|_{T,V}
\end{equation*}

\end{document}