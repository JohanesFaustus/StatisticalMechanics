\documentclass[../../../main.tex]{subfiles}
\begin{document}
Solids that can be described as crystal are called polycrystalline; these material are actually composed of small crystalline grains.
This collection of crystal--or grains--have their own orientation.
The interface of two grains with different orientation are called grain boundary and this can often make the crystal have more mechanical strength.

On the other hand, solids that  characterized by the absence of any long-range order are called amorphous. 
Without order, their atomic structure cannot form grain, thus they do not have grain boundary.

\subsection{Mathematical Description}
\subsubsection{Bravais lattice.} 
Defined as an infinite array of discrete points with identical surroundings. 
These discrete points are generated by the primitive generating vector
\begin{equation*}
    \mathbf{R} =m \mathbf{a }_1+n \mathbf{a }_2 +o \mathbf{a }_3
\end{equation*}
If we were to consider this definition only, we will have the definition of mathematical lattice.
The Bravais lattice then put another restriction: any lattice point can be chosen as the origin, and the set of relative positions of all other lattice points remains exactly the same.
The equivalence of all point, in other words.

In the case of simple 2D lattice, the magnitude of primitive vector $\mathbf{a }_n$ is the lattice constant.
However, for 3D lattice, the primitive vector $\mathbf{a }_n$ do not align with the cube edges--for the case of FCC and BCC.
In this case, the lattice constant is defined as the length of the cube edges.

\subsubsection{Unit cells.}
Defined as any volume of space when translated through all the vector in Bravais lattice, fills space without overlap and without leaving voids.
Primitive unit cells contain only one lattice point, whereas nonprimitive unit cells contain more than  one unit cells.
A special choice of the primitive unit cell is the
Wigner-Seitz cell that is defined as the closest region of space to one given lattice point than to any other.
\begin{table}[h]
    \centering
    \caption{Difference of primitive and nonprimitive unit cells}
    \begin{tabular}{ cc>{\centering\arraybackslash}p{2cm}} 
        \toprule
        &Primitive&Nonprimitive\\
        \midrule
        Volume&Smaller&Bigger\\
        Lattice points&One&More than one\\
        Function&Standard definition&Structure representation--such as FCC, HCP, and BCC--whose primitive unit cells can be derived\\
        \bottomrule
    \end{tabular}
\end{table}
\begin{figure*}[]
    \centering
    \normfig{../../../Rss/SolidState/Crystal/unitcells.jpg}
    \caption*{Figure: Primitive and nonprimitive unit cells}
\end{figure*}

\subsubsection*{Basis.}
Defined as a set of one or more atoms associated with each lattice point.
Using both Bravais lattice and basis, the crystal structure can be described.
The position of all atom--also called crystal points--is written as 
\begin{equation*}
    \text{Crystal points}=\mathbf{R}+\mathbf{r}_i
\end{equation*}
with $\mathbf{r}_i$ as the position of the $i$-th atom in the basis relative to that lattice points.
\begin{figure*}[h]
    \centering
    \normfig{../../../Rss/SolidState/Crystal/basis.jpeg}
    \caption*{Figure: Basis}
\end{figure*}
\end{document}