\documentclass[../../../main.tex]{subfiles}
\begin{document}
Solids that can be described as crystal are called polycrystalline; these material are actually composed of small crystalline grains.
This collection of crystal--or grains--have their own orientation.
The interface of two grains with different orientation are called grain boundary and this can often make the crystal have more mechanical strength.

On the other hand, solids that  characterized by the absence of any long-range order are called amorphous. 
Without order, their atomic structure cannot form grain, thus they do not have grain boundary.

\section{Mathematical Description}
\subsection{Bravais lattice.} 
Defined as an infinite array of discrete points with identical surroundings. 
These discrete points are generated by the primitive generating vector
\begin{equation*}
    \mathbf{R} =m \mathbf{a }_1+n \mathbf{a }_2 +o \mathbf{a }_3
\end{equation*}
If we were to consider this definition only, we will have the definition of mathematical lattice.
The Bravais lattice then put another restriction: any lattice point can be chosen as the origin, and the set of relative positions of all other lattice points remains exactly the same.
The equivalence of all point, in other words.

In the case of simple 2D lattice, the magnitude of primitive vector $\mathbf{a }_n$ is the lattice constant.
However, for 3D lattice, the primitive vector $\mathbf{a }_n$ do not align with the cube edges--for the case of FCC and BCC.
In this case, the lattice constant is defined as the length of the cube edges.


\end{document}